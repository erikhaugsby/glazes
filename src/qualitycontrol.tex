\chapter{Quality Control}
A successful business depends on consistent results. This can only be done if 
quality control is made a habit. This means having regular procedures for 
storing glaze materials, checking new shipments, weighing, grinding, mixing, 
and checking each new batch of glaze before using it in production.
%-------------------------------------------------------------------------------
\section{Raw Materials Control}
Raw materials suppliers have their own problems with getting consistent 
materials. Sometimes they may send you a different material without any 
notification, or the quality of material from the mine may change. If you have 
enough working capital and storage area, it is best to get raw materials in 
large quantities, up to one year's need.
%-------------------------------------------------------------------------------
\subsection{Raw Material Testing}
When you get a new shipment of raw materials, each one should be tested. For 
the individual potter this is simply done by testing each one in the standard 
glaze recipe to see if there is any change. About 200 grams of glaze is mixed 
using the old stock materials and replacing only one of the new materials at a 
time. If the test glaze is different from your standard glaze, it will be 
necessary to alter your glaze recipe.
%-------------------------------------------------------------------------------
\subsection{Storing of Glaze Materials}
All materials should be kept in bags or buckets so there is no chance of mixing 
up different materials. Mark the contents of all bags and buckets and the 
material's delivery date on labels that cannot easily be removed. Keep the 
glaze material store separated from working areas and make sure that only 
responsible persons have access to it.
%-------------------------------------------------------------------------------
\section{Glaze Preparation Control}
Many glaze problems are caused by carelessness during mixing of the glaze. When 
preparing glazes and frit, make sure that the right recipe is used and that the 
weighing is done correctly.
%-------------------------------------------------------------------------------
\subsection{Batch Cards}
If you are running a small pottery and you are doing all glaze work yourself, 
you can rely on a very simple system. Still, write down your recipe, keep it 
next to the balance and after weighing each material tick it off on the recipe.

For larger productions use a batch card system. A batch card form is shown in 
section~\ref{sec:batchcard}. The card follows the glaze batch during its 
preparation and later when the glaze is used in production. It has three 
purposes:
%-------------------------------------------------------------------------------
\begin{itemize}
\item It shows the glaze mixer the recipe, ball milling time, density of the 
glaze slip.
\item The supervisor can easily check if all instructions are followed.
\item If something goes wrong, the batch card helps to trace the cause of the 
problem.
\end{itemize}
%-------------------------------------------------------------------------------
The batch card number should be marked on the glaze bucket. To avoid mistakes 
tie a tile glazed with the same glaze to the bucket.
%-------------------------------------------------------------------------------
\subsection{Balance}
The balance and the weights need to be checked now and then. The weights should 
be clean. The balance may become inaccurate because the scales get dirty or the 
pivots or beams get out of alignment. After cleaning the weights they and the 
balance are checked by weighing something with a known weight ( 1 liter of 
water weighs 1 kg).
%-------------------------------------------------------------------------------
\subsection{Graduated Cylinder}
Cylinders or flasks used for measuring volume are used for adjusting density of 
glaze slips. Unfortunately, measuring cylinders are often not graduated 
correctly by the manufacturer. The cylinder can be checked by filling it with 
water to its mark, say 250 ml and then checking if the water weighs 250 g. In 
some cases they have been out by more than 10\%.
%-------------------------------------------------------------------------------
\subsection{Ball Milling}
The fineness of the glaze particles influences the glaze very much. To keep 
this constant, make sure that the ball milling time is the same. The time 
should be noted on the batch card. If different glazes are milled in the same 
ball mill, the worker must enter on the card that he has cleaned the ball mill 
before loading it. The supervisor should check that the ball mill lining and 
pebbles are correct.
%-------------------------------------------------------------------------------
\subsection{Sieving}
The glaze should be screened before use. On the batch card screen mesh size is 
mentioned. Check the residue on the screen. If you get more residue than usual, 
there may be something wrong with the ball milling.
%-------------------------------------------------------------------------------
\section{Methods of Testing Batches of Glaze and Frits}
\subsection{Testing Frit}
Molten frit can be drawn from the frit kiln to see whether all ingredients are 
well melted and whether air bubbles are released. Air bubbles may not be a 
problem, since many of them will be released during grinding and the second 
glaze firing. But if air bubbles (pinholes) give trouble during glaze firing, 
it may be a good idea to extend fritting time, so that the air has time to 
escape. In continuous frit kilns, bars of refractory brick can be placed on the 
sloping floor to slow down the flow of frit.

After fritting is over, the melting temperature and the viscosity of the frit 
can be compared with previous batches of frit by melting a fixed amount of frit 
on a sloped tile.
%-------------------------------------------------------------------------------
\subsection{Testing Glaze}
Each new batch of glaze should be made at least one firing before using it. 
This will give enough time to apply the glaze to a few test pieces and fire 
them in the regular glaze firing. Glaze at least three pieces and place one in 
a cold spot, one in a normal and one in a hot spot. If something is wrong with 
the glaze, this will prevent a whole kilnload from being ruined.
%-------------------------------------------------------------------------------
