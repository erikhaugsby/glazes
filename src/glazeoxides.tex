\chapter{Glaze Oxides}
\label{sec:glazeoxides}
%-------------------------------------------------------------------------------
This list of glaze materials only includes materials that we can expect to 
obtain easily. It also lists mineral sources for the oxides, which may be 
helpful when consulting with geologists. The formula suggestions are only meant 
as a rough guide for those who work with formulas.

The following list can be used as a reference when developing or modifying 
glazes. If you have a problem with pinholes, then go through the list noting 
down all materials that are mentioned as having high viscosity or high surface 
tension or the opposite. Comparing your glaze recipe with your notes will give 
ideas of what materials to increase or decrease.

\textbf{Note:} ``MP'' means ``melting point''.
%-------------------------------------------------------------------------------
\section{Aluminum oxide}
Alumina, \ce{Al2O3}; stabilizer; melting point=2050\degree C

Sources:
\begin{itemize}
\item Aluminum oxide, alumina, \ce{Al2O3}
\item Clay, \ce{Al2O3*2SiO2*2H2O}
\item Feldspar, \ce{K2O/N2O/CaO*Al2O3*6SiO2}
\item Mineral sources: kaolin, ball clay, bentonite, corundum, bauxite, 
silimanite, kyanite, gibbsite (hydrargillite), websterite (aluminite), alunogen.
\end{itemize}
%-------------------------------------------------------------------------------
Effect:
%-------------------------------------------------------------------------------
\begin{itemize}
\item Increases melting point, hardness, viscosity, surface tension.
\item Reduces tendency of crystal formation.
\item Reduces thermal expansion.
\item Small additions help other opacifiers.
\item Large amounts produce matt glazes.
\end{itemize}
%-------------------------------------------------------------------------------
Clay addition normally 5--15\%. 

Clay helps to keep glaze materials suspended in 
the bucket. Large additions cause problems of cracking of raw glaze layer and 
crawling, pinholing.

Ratios:
%-------------------------------------------------------------------------------
\begin{itemize}
\item Shiny glazes: $Alumina:Silica = 1:6-1:10$
\item Matt glazes: $Alumina:Silica = 1:4-1:2$
\end{itemize}
%-------------------------------------------------------------------------------
\section{Barium oxide}
Baria, \ce{BaO}; flux; melting point=1923\degree C

Sources:
\begin{itemize}
  \item Barium carbonate, \ce{BaCO3}, poisonous if enters blood
  \item Barium sulfate, \ce{BaSO4}
  \item Selenite, \ce{BaO*SeO2}
  \item Mineral sources: witherite, barytes, celsian, bromlite, barytocalcite.
\end{itemize}
%-------------------------------------------------------------------------------
Effect:
%-------------------------------------------------------------------------------
\begin{itemize}
\item Reduces boron's tendency to form opaque ``clouds'' and therefore helps to 
make boron glaze transparent.
\item Reduces chemical resistance.
\item High amounts (above 25\%) produce matt glaze due to formation of 
crystals. \ce{BaO} matt glazes are not stable.
\item Lowers melting point.
\item Slow in giving off \ce{CO2}. Sometimes sulfate problems in coal-or 
oil-fired kilns.
\item Helps formation of crystalline glazes.
\item Improves hardness.
\item Small amounts improve gloss.
\end{itemize}
%-------------------------------------------------------------------------------
Formula:
%-------------------------------------------------------------------------------
\begin{itemize}
  \item Generally, below 1100\degree C \ce{BaO} should be less than 0.10 mole.
\item Above 0.3 mole \ce{BaO} raises melting point of glaze.
\end{itemize}
%-------------------------------------------------------------------------------
Color effect:
%-------------------------------------------------------------------------------
\begin{itemize}
\item \ce{CoO} colors turn more violet, \ce{Cr2O3} below 1\% turns more yellow.
\item \ce{CuO} colors turn from green to blue-green.
\item Iron colors are subdued.
\item \ce{NiO} colors turn more brownish.
\end{itemize}
%-------------------------------------------------------------------------------
\section{Boric oxide}
\ce{B2O3}; stabilizer or glass former; melting point=741\degree C

Boric oxide is sometimes classified as a stabilizer (USA) and sometimes as a 
glass former (UK).

Sometimes a small percentage of raw borax is added to glaze or to engobe. When 
the glaze layer dries, the borax recrystallizes and this gives strength to the 
raw glaze layer which means it will not be damaged during handling.

Sources:
%-------------------------------------------------------------------------------
\begin{itemize}
\item Borax (\ce{Na2B4O7*10H2O})
\item Boric acid (\ce{B2O3*3H2O})
\item Both materials are soluble in water and they are normally introduced in a 
frit.
\item Colemanite, gerstley borate (\ce{2CaO*3B2O3*5H2O}). The only insoluble 
mineral form of borax, only mined in the USA.
\item Calcium borate (\ce{CaO*B2O3*6H2O2}), the chemical form of colemanite.
\item Mineral sources: Borax (tincal), kernite, ulexite, colemanite, boracite, 
sassolin.
\end{itemize}
%-------------------------------------------------------------------------------
Effect:
%-------------------------------------------------------------------------------
\begin{itemize}
  \item Strongly lowers melting point. Mainly used below 1100\degree C.
  \item Improves formation of an intermediate layer between glaze and body.
  \item Boric oxide below 15\% reduces tendency to craze, higher amounts 
  increase crazing.
  \item Lowers viscosity and surface tension.
  \item Low thermal expansion rate.
  \item \ce{B2O3} less than 10\% lowers surface tension.
  \item High content of boric oxide forms opaque clouds especially in 
  combinations with \ce{CaO} and \ce{SnO2}. This is reduced by addition of 
  \ce{BaO} or \ce{SrCO3}.
  \item Extends the firing range.
  \item Reduces tendency to crystallize.
\end{itemize}
%-------------------------------------------------------------------------------
Formula: Boric oxide ratio to silica is normally 1:10 and should not be less 
than 1:2. In frits a ratio below 1:2 will leave the frit water-soluble.

Color effect:
%-------------------------------------------------------------------------------
\begin{itemize}
\item MnO colors turn a violet hue.
\item Iron colors become yellowish-reddish.
\item CoO colors become brighter.
\item CuO colors change from green to bluish green.
\end{itemize}
%-------------------------------------------------------------------------------
\section{Calcium oxide}
Calcia, \ce{CaO}; flux; melting point=2570\degree C

Sources:
\begin{itemize}
  \item Calcium carbonate (\ce{CaCO3}), limestone, whiting, marble.
  \item Wollastonite (\ce{CaO*SiO2}).
  \item Dolomite (\ce{CaCO3*MgCO3}).
  \item Anorthite, lime feldspar (\ce{CaO*Al2O3*2SiO2}).
  \item Calcium sulfate (\ce{CaSO4}), plaster of paris.
  \item Calcium borate (\ce{2CaO*3B2O3*5H2O}).
  \item Calcium fluoride (\ce{CaF2}).
  \item Calcium phosphate (\ce{Ca3(PO4)2}) (bone ash).
  \item Mineral sources: glauberite, fluorspar, apatite, lime, calcite, chalk, 
  limestone, marble, gypsum, alabaster, seashells, coral, portland cement.
\end{itemize}
%-------------------------------------------------------------------------------
Effect:
%-------------------------------------------------------------------------------
\begin{itemize}
  \item Combines readily with silica in glaze and, if CaO is present in body, 
  it reacts with \ce{SiO2} in glaze to form a strong interface, reducing 
  crazing.
 \item Increases hardness, especially with boron glazes.
 \item Reduces tendency to craze.
 \item Primary flux for temperatures above 1100\degree C.
 \item Below 1100\degree C small additions act as secondary flux.
 \item High CaO produces opacity in boron glazes, and white matt wax-like 
 glazes can be produced.
 \item Too high CaO gives dull, matt finish.
 \item \ce{CaCO3} gives off \ce{CO2} at 825\degree  C.
 \item In zircon white glaze CaO increases pinholes and a dull surface.
 \item Decreases lead solubility.
  
\end{itemize}
%-------------------------------------------------------------------------------
Ratios:
%-------------------------------------------------------------------------------
\begin{itemize}
  \item At cone 03 CaO not above 0.25-0.28 mole
  \item At cone 01 CaO not above 0.30-0.35 mole
\end{itemize}
%-------------------------------------------------------------------------------
Color effect:
%-------------------------------------------------------------------------------
\begin{itemize}
  \item CaO turns \ce{Cr2O3} colors yellow.
  \item MnO browns and violets are improved with CaO.
  \item CaO is important for production of iron-red, chrome-green and blue 
  color pigments.
\end{itemize}
%-------------------------------------------------------------------------------
\section{Lead oxide}
\ce{PbO}; flux; melting point=888\degree C

Lead is a very good flux but it is very poisonous and expensive. It should 
never be used in ware that will contain food, but still is used frequently for 
decorative ware. If you use lead, it should always be in frit form.

Sources:
\begin{itemize}
\item Litharge (\ce{PbLO})
\item Red lead (\ce{Pb3O4})
  \item White lead, lead carbonate (\ce{2PbCO3*Pb(OH)2})
  \item Mineral sources: glauberite, fluorspar, apatite, lime, calcite, chalk, 
  limestone, marble, gypsum, alabaster, seashells, coral, portland cement.
\end{itemize}
%-------------------------------------------------------------------------------
Effect:
%-------------------------------------------------------------------------------
\begin{itemize}
  \item Smooth, shiny low-temperature glazes.
  \item Strong flux.
  \item Good for transparent glazes.
  \item Reduces viscosity and surface tension.
  \item Reduces hardness and chemical resistance.
  \item Evaporates easily during firing.
  \item Combined with boric oxide, it is a common flux for earthenware glazes.
  \item It is more dangerous with copper oxide, which increases lead release 10 
  times.
  \item Small amounts in high temperature increase smoothness.
\end{itemize}
%-------------------------------------------------------------------------------
Ratios:

Simple lead-alumina-silicate combinations make glazes in ratios as shown in 
table~\ref{tab:formulaleadglaze}.
%-------------------------------------------------------------------------------
\begin{center}
  \renewcommand{\arraystretch}{1.5}
  \begin{table}\centering
    \begin{tabular}{|c|c|c|c|}\hline
      \textbf{Temperature}&\textbf{Lead}&\textbf{Alumina}&\textbf{Silica}\\\hline\hline
      900\degree C&0.10&1.0&1\\\hline
      920\degree C&0.11&1.1&1\\\hline
      940\degree C&0.12&1.2&1\\\hline
      960\degree C&0.13&1.3&1\\\hline
      980\degree C&0.14&1.4&1\\\hline
      1000\degree C&0.15&1.5&1\\\hline
      \vdots&\vdots&\vdots&\vdots\\\hline
      1200\degree C&0.25&2.5&1\\\hline
     \end{tabular}
    \caption{Simple lead-alumina-silicate glaze combinations.}
    \label{tab:formulaleadglaze}
  \end{table}
\end{center}
%-------------------------------------------------------------------------------
Color effect:
%-------------------------------------------------------------------------------
\begin{itemize}
\item Good with almost all colorants.
\item Lead transparent glazes produce pleasant colors for engobe decorations.
\item With iron, rich tans, browns, reds.
\item With copper, rich greens (\textbf{Caution:} Lead release is increased 10 
times).
\item With antimony oxide, yellow.
\end{itemize}
%-------------------------------------------------------------------------------
\section{Lithium oxide}
\ce{Li2O}; flux; melting point\textgreater 618\degree C

High price. A number of artificial lithium chemicals exist.

Sources:
\begin{itemize}
  \item Lepidolite (lithium mica), 1.5--6\% lithium oxide.
  \item Petalite (\ce{Li2O*Al2O3*SiO2}), 2--4\% lithium oxide.
  \item Spodumene (\ce{Li2O*Al2O3*4SiO2}), about 8\% lithium oxide.
  \item Lithium carbonate (\ce{Li2CO3}).
\end{itemize}
%-------------------------------------------------------------------------------
Effect:
%-------------------------------------------------------------------------------
\begin{itemize}
  \item A strong flux
  \item Lowers viscosity.
  \item Improves hardness.
  \item Improves gloss.
  \item High \ce{Li2O} content furthers formation of crystals in the melted 
  glaze.
  \item Additions of \ce{Li2CO3} as low as 1\% improve gloss and smoothness of 
  glaze.
\end{itemize}
%-------------------------------------------------------------------------------
Color effect:
%-------------------------------------------------------------------------------
\begin{itemize}
  \item CuO turns to blue colors.
  \item In lithium glaze 1\% \ce{SnO2} + 0.5\% CuO produces Chinese reds in 
  reduction firings.
\end{itemize}
%-------------------------------------------------------------------------------
\section{Magnesium oxide}
Magnesia, \ce{MgO}; flux; melting point=2800\degree C

Sources:
\begin{itemize}
  \item Talc (\ce{3MgO*4SiO2*H2O}).
  \item Magnesite (magnesium carbonate) (\ce{MgCO3})
  \item Dolomite (\ce{CaCO3*MgCO3})
  \item Mineral sources: Soapstone or steatite, serpentine, meerschaum, 
  vermiculite, periclase magnesia, magnesite, brucite.
\end{itemize}
%-------------------------------------------------------------------------------
Effect:
%-------------------------------------------------------------------------------
\begin{itemize}
  \item Raises melting point.
  \item High surface tension.
  \item Reduces crazing due to its low thermal expansion.
  \item Small amounts increase gloss.
  \item Larger amounts make matt glaze (best above 1100\degree C).
  \item With double glazing, good for special-effect crawling glaze.
\end{itemize}
%-------------------------------------------------------------------------------
Ratios: Below 1100\degree C, less than 0.1 mole MgO increases gloss and 
0.2--0.4 mole MgO produces matt glazes.
%-------------------------------------------------------------------------------
Color effect:
%-------------------------------------------------------------------------------
\begin{itemize}
  \item CoO blue turns violet with MgO.
  \item MgO glaze on red iron rich body turns the red color to a dirty 
  yellow-brown color. Therefore transparent glaze should contain no MgO.
  \item \ce{Cr2O3} green only accepts small amounts of MgO. Large amounts 
  bleach the green color.
\end{itemize}
%-------------------------------------------------------------------------------
\section{Phosphorous oxide}
\ce{P2O5}; glass former; melting point=569\degree C

Sources:
\begin{itemize}
  \item Bone ash, calcium phosphate (\ce{Ca3(PO4)2}).
  \item Apatite, \ce{3Ca3(PO4)2Ca(ClF)2}
  \item Mineral sources: Bone ash (made from calcining animal bones), apatite, 
  wavellite, vivianite.
\end{itemize}
%-------------------------------------------------------------------------------
Effect:
%-------------------------------------------------------------------------------
\begin{itemize}
  \item \ce{P2O5} can replace some of the \ce{SiO2} in the glaze.
  \item Strong flux, especially with MgO, BaO and alkalis.
  \item Additions above 5\% form opaque glaze, especially in combination with 
  ZnO and in lead-free glazes.
  \item Additions of up to 4\% may increase melting and reduce pinholes. 
  However, bone ash often increases pinholes due to high release of gas 
  (instead add the bone ash to the frit).
  \item High additions (above 10\%) produce matt glaze.
  \item Additions above 25\%--30\% make the glaze too soluble (less acid-or 
  weather-resistant).
\end{itemize}
%-------------------------------------------------------------------------------
Color effect:
%-------------------------------------------------------------------------------
\begin{itemize}
  \item CoO blue turns more violet.
  \item In \ce{B2O3} glazes iron colors turn yellowish.
  \item In alkaline glazes iron colors turn white with high amount of \ce{P2O5}.
  \item CuO greens turn bluish and, with high \ce{P2O5}, spotted.
  \item MnO colors turn more violet.
  \item \ce{Cr2O3} colors are improved to lighter shades.
  \item Interesting special surface effects with high \ce{P2O5}.
\end{itemize}
%-------------------------------------------------------------------------------
\section{Potassium oxide}
Potash, \ce{K2O}; flux; melting point=896\degree C

Sources:
\begin{itemize}
  \item Potassium carbonate, potash (pearl ash) (\ce{K2CO3}), water-soluble.
  \item Potassium nitrate, saltpeter (\ce{KNO3}), water-soluble--also used as 
  fertilizer.
  \item Potash feldspar (\ce{K2O*Al2O3*6SiO2}), exists as minerals named 
  orthoclase and microcline, melting at 1200\degree C.
  \item Nepheline syenite (\ce{3Na2O*K2O*4Al2O*8SiO2}).
  \item Mineral sources: saltpeter, potassium bichromate, leucite.
\end{itemize}
%-------------------------------------------------------------------------------
Effect:
%-------------------------------------------------------------------------------
\begin{itemize}
  \item Potash's effect is very similar to soda, but it is a slightly less 
  powerful flux.
  \item Potash increases crazing, but a little less than soda does.
\end{itemize}
%-------------------------------------------------------------------------------
\section{Silicon oxide}
Silica, \ce{SiO2}; glass former; melting point=1710\degree C

Sources:
\begin{itemize}
  \item Quartz, \ce{SiO2}
  \item Clay, \ce{Al2O3*2SiO2*2H2O}
  \item Feldspar, \ce{Na2O/K2O/CaO*Al2O3*6SiO2}
  \item Talc, \ce{3MgO*4SiO2*H2O}
  \item Zirconium silicate, \ce{ZrSiO4}
  \item Wollastonite, \ce{CaO*SiO2}
  \item Mineral sources: flint, chalcedony, chert, sand, quartzite, diatomite, 
  granite, part of all rocks.
\end{itemize}
%-------------------------------------------------------------------------------
Effect:
%-------------------------------------------------------------------------------
\begin{itemize}
  \item A glass former, a part of all glazes.
  \item Generally raises melting temperature.
  \item Low thermal expansion, addition reduces crazing.
  \item Addition to body also reduces crazing (see glaze faults).
  \item Increases viscosity of glaze melt.
  \item Increases acid and weather resistance.
  \item Increases hardness of glaze.
  \item High amounts make the glaze shiver.
\end{itemize}
%-------------------------------------------------------------------------------
Ratios:
%-------------------------------------------------------------------------------
\begin{itemize}
  \item Addition of 0.1 mole \ce{SiO2} increases melting point by 20\degree C.
  \item Amount of \ce{SiO2} depends on other glass-forming oxides. In general, 
  earthenware: 1--2.5 mole \ce{SiO2} and stoneware: 1--4 mole \ce{SiO2}.  
\end{itemize}
%-------------------------------------------------------------------------------
\section{Sodium oxide}
Soda, \ce{Na2O}; flux; melting point$\approx$800\degree C

Sources:
\begin{itemize}
  \item Sodium carbonate (\ce{Na2CO3}) as crystal soda or calcined soda, also 
  named soda ash, soluble in water, absorbs moisture from the air.
  \item Sodium nitrate (\ce{NaNO3}). Sodium saltpeter (Chile saltpeter), 
  soluble in water.
  \item Sodium chloride (\ce{NaCl}). Table salt, water-soluble, used in salt 
  glazing, used in frit for reducing discoloration of frit by iron compounds.
  \item Soda feldspar or albite (\ce{Na2O*Al2O*6SiO2}), a white mineral melting 
  at 1170\degree C.
  \item Nepheline syenite, (\ce{K2O*3Na2O*4Al2O3*8SiO2}), mineral melting at 
  1100\degree --1200\degree C
  \item Mineral sources: natron, halite, hauynite, plagioclase, oligoclase, 
  sodalite, glauberite, cryolite, glauber salt.
\end{itemize}
%-------------------------------------------------------------------------------
Effect:
%-------------------------------------------------------------------------------
\begin{itemize}
  \item Strong fluxing agent.
  \item Improves gloss.
  \item Very high thermal expansion induces crazing.
  \item Lowers elasticity of glaze, which becomes brittle with high amount of 
  \ce{Na2O}.
  \item Low viscosity, causes glaze to run. Short melting range.
  \item Evaporates easily above 1100\degree C (salt glazing).
\end{itemize}
%-------------------------------------------------------------------------------
Ratios: in alkaline frits 1 mole alkali with at least 2.5 mole \ce{SiO2}, 
otherwise the alkalis \ce{Na2O} and \ce{K2O} will remain water-soluble.
%-------------------------------------------------------------------------------
Color effect:
%-------------------------------------------------------------------------------
\begin{itemize}
  \item - High amount of \ce{Na2O} or \ce{K2O} produces ``alkaline colors'', 
  noted for their brightness and interesting shades.
  \item Copper oxide turns blue instead of green.
  \item Manganese oxide turns violet.
  \item Cobalt gives a light blue.
  \item Iron oxide produces red in connection with boron.
\end{itemize}
%-------------------------------------------------------------------------------
\section{Tin oxide}
\ce{SnO2}; glass-former; melting point=1930\degree C

Sources:
\begin{itemize}
  \item Tin oxide, \ce{SnO2} (artificial)
  \item Mineral sources: cassiterite (tinstone), stannite, tin pyrites.
\end{itemize}
%-------------------------------------------------------------------------------
Effect:
%-------------------------------------------------------------------------------
\begin{itemize}
  \item Opacifier with 5--10\% addition, less efficient in alkali-rich glazes.
  \item Opacifying effect increases with CaO, \ce{TiO2} and \ce{ZrO2}. Fine 
  grinding improves opacifying effect.
  \item Increases viscosity and melting point.
  \item Increases hardness and acid resistance.
  \item Increases elasticity of glaze (reduces crazing).
\end{itemize}
%-------------------------------------------------------------------------------
Color effect:
%-------------------------------------------------------------------------------
\begin{itemize}
  \item In leadless glaze turns CuO bluish.
  \item Produces pink in combination with \ce{Cr2O3} and CaO.
  \item Iron brown colors turn redder.
  \item Manganese brown turns more violet.
  \item Used for stabilizing colors in pigment production.
\end{itemize}
%-------------------------------------------------------------------------------
\section{Titanium dioxide}
\ce{TiO2}; glass-former; melting point=1855\degree C

Sources:
\begin{itemize}
  \item Titanium dioxide, titania, \ce{TiO2} (artificial)
  \item Rutile, \ce{TiO2} (85--98\% \ce{TiO2})
  \item Perovskite, \ce{CaO*TiO2}
  \item Titanite, \ce{CaO*TiO2*SiO2}
  \item Ilmenite, \ce{FeO*TiO2}
  \item Mineral sources: rutile, anatase, brookite, titanite (sphere), ilmenite.
\end{itemize}
%-------------------------------------------------------------------------------
Effect:
%-------------------------------------------------------------------------------
\begin{itemize}
  \item Opacifier but not so reliable. Opacity improves with addition of ZnO 
  and CaO.
  \item Above 10\% \ce{TiO2}, glaze turns matt due to forming of small crystals 
  if cooling is slow. Mattness depends very much on firing conditions.
  \item Reduces crazing.
  \item Increases acid resistance.
  \item Reduces lead solubility when introduced in small amounts.
  \item Used for crystal glazes in combination with ZnO.
\end{itemize}
%-------------------------------------------------------------------------------
Color effect:
%-------------------------------------------------------------------------------
\begin{itemize}
  \item Pure \ce{TiO2} produces white colors in alkali-rich, lead-free glazes.
  \item In lead glazes and high boron glazes with small amounts of iron oxide a 
  slight yellow color is obtained.
  \item Rutile contains some iron. The pure \ce{TiO2} will work as rutile with 
  an addition of about 5\% iron oxide.
  \item On iron-rich bodies (red firing) \ce{TiO2} combines with the iron of 
  the body to form yellow-brown colors.
  \item \ce{TiO2} addition turns CoO blue to gray-blue and with high CoO to 
  green.
  \item Low CuO turns yellowish, high CuO bluish.
  \item \ce{Cr2O3} becomes dirty greyish.
  \item \ce{MnO2} turns greyish
  \item NiO red and blue colors changed to green.
\end{itemize}
%-------------------------------------------------------------------------------
\section{Zinc oxide}
\ce{ZnO}; flux; melting point=1975\degree C

Sources:
\begin{itemize}
  \item Zinc oxide, zinc white, ZnO
  \item Zinc borate, \ce{ZnO*B2O3}
  \item Zinc chloride, \ce{ZnCl2}
  \item Zinc phosphate, \ce{3ZnO*P2O5}
  \item Mineral sources: sphalerite or blende (zinc sulfide), the original zinc 
  ore, smithsonite, hydrozincite, willemite.
\end{itemize}
%-------------------------------------------------------------------------------
Effect:
%-------------------------------------------------------------------------------
\begin{itemize}
  \item Above 1100\degree C a strong flux.
  \item In small amounts increases brilliance.
  \item High amounts produce matt glazes.
  \item Reduces viscosity, increases surface tension.
  \item Increases boron clouds and helps opacity in combination with other 
  opacifiers.
  \item Reduces crazing due to its low thermal expansion and high elasticity.
  \item Its high drying shrinkage may cause crawling if added without prior 
  calcination.
  \item In high amounts best agent for forming crystals.
  \item Produces special surface and color effect in high boron glazes.
\end{itemize}
%-------------------------------------------------------------------------------
Color effect:
%-------------------------------------------------------------------------------
\begin{itemize}
  \item Generally increases brightness of colors.
  \item Chrome-green turns gray.
  \item Cobalt blue becomes lighter with less of a violet hue.
  \item Manganese violet turns brown.
\end{itemize}
%-------------------------------------------------------------------------------
\section{Zirconium oxide}
\ce{ZrO2}; glass-former; melting point=2700\degree C

Sources:
\begin{itemize}
  \item Zircon, zirconium silicate, \ce{ZrSiO4}
  \item Zirconium oxide, zirconia, \ce{ZrO2}
  \item Commercial opacifiers
  \item Commercial zircon frits.
  \item Mineral sources: beach sands, baddeleyite (\ce{ZrO2}).
\end{itemize}
%-------------------------------------------------------------------------------
Effect:
%-------------------------------------------------------------------------------
\begin{itemize}
  \item Zircon additions of 10--20\% produce opaque white glaze (due to its 
  high price zirconium oxide is seldom used).
  \item Used in combination with ZnO, MgO, BaO, \ce{SnO2} opacity is increased.
  \item Opacity is furthered by fine grinding and by adding zircon to the frit 
  instead of the batch.
  \item Increases melting point.
  \item Increases hardness, viscosity and surface tension.
  \item Increases tendency to form pinholes.
  \item Reduces crazing.
\end{itemize}
%-------------------------------------------------------------------------------