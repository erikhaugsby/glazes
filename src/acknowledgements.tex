\chapter*{Acknowledgements}
The original book is entitled \textit{Glazes--for the Self-Reliant Potter}, 
published in 1993.

DOI: 10.1007/978-3-663-06865-5

ISBN: 9783663068655 (online) 9783528020675 (print)

This document was compiled by Erik Haugsby from the source material provided 
via the CD3WD:
%-------------------------------------------------------------------------------
\begin{verbatim}
http://www.fastonline.org/CD3WD_40/CD3WD/APPRTECH/G17GLE/EN/B483.HTM
\end{verbatim}
%-------------------------------------------------------------------------------
I have made all attempts to preserve the original text and formating. The 
possibility of minor or unavoidable changes in layout and formatting, as well 
as unintentional errors in transcribing the original text, cannot be excluded.

Should you find any errors or omissions, I would greatly appreciate you either 
notifying me \href{mailto:e@erikhaugsby.com}{by email}, or initiating a pull 
request via Git.
%-------------------------------------------------------------------------------
\section*{The authors:}
%-------------------------------------------------------------------------------
\subsection*{Henrik Norsker} 
Henrik Norsker has been making pottery since 1970. He left his 
pottery workshop in Denmark in 1976 to establish a pottery school in a village 
in Tanzania. Since then he has continued working in developing countries with 
promotion of small scale ceramics industries. Besides Tanzania he has been 
employed in ceramics projects in Burma, Bangladesh and Nepal.
%-------------------------------------------------------------------------------
\subsection*{James Danisch} 
James Danisch has been making, selling and experimenting with 
ceramics since 1963. He has taught college level ceramics in Scotland and 
California, and has conducted workshops in the US, South America and Canada. 
From 1984 to 1992, he has been working with small scale and rural ceramics 
development in Nepal. His articles on ceramics have been published in several 
magazines, and he has studied traditional and he has studied traditional and 
modern techniques in Europe, Nepal, India, Thailand, Burma, South America and 
Mexico.
%-------------------------------------------------------------------------------
\subsection*{The Deutsches Zentrum f\"{u}r Entwicklungstechnologien}
Deutsches Zentrum f\"{u}r Entwicklungstechnologien-GATE

Deutsches Zentrum f\"{u}r Entwicklungstechnologien--GATE--stands for German 
Appropriate Technology Exchange. It was founded in 1978 as a special division 
of the Deutsche Gesellschaft f\"{u}r Technische Zusammenarbeit (GTZ) GmbH. GATE 
is a 
centre for the dissemination and promotion of appropriate technologies for 
developing countries. GATE defines ``Appropriate technologies'' as those which 
are suitable and acceptable in the light of economic, social and cultural 
criteria. They should contribute to socio-economic development whilst ensuring 
optimal utilization of resources and minimal detriment to the environment. 
Depending on the case at hand a traditional, intermediate or highly-developed 
can be the ``appropriate'' one. GATE focusses its work on the key areas:
%-------------------------------------------------------------------------------
\begin{itemize}
\item Dissemination of Appropriate Technologies: 

Collecting, processing and disseminating information on technologies 
appropriate to the needs of the developing countries: ascertaining the 
technological requirements of Third World countries: support in the form of 
personnel, material and equipment to promote the development and adaptation of 
technologies for developing countries.

\item Environmental Protection:

The growing importance of ecology and environmental protection require better 
coordination and harmonization of projects. In order to tackle these tasks more 
effectively, a coordination center was set up within GATE in 1985.
\end{itemize}
%-------------------------------------------------------------------------------
GATE has entered into cooperation agreements with a number of technology 
centres in Third World countries.

GATE offers a free information service on appropriate technologies for all 
public and private development institutions in developing countries, dealing 
with the development, adaptation, introduction and application of technologies.

Deutsche Gesellschaft f\"{u}r Technische Zusammenarbeit (GTZ) GmbH

The government-owned GTZ operates in the field of Technical Cooperation. 2200 
German experts are working together with partners from about 100 countries of 
Africa, Asia and Latin America in projects covering practically every sector of 
agriculture, forestry, economic development, social services and institutional 
and material infrastructure. The GTZ is commissioned to do this work both by 
the Government of the Federal Republic of Germany and by other government or 
semi-government authorities.

The GTZ activities encompass:
%-------------------------------------------------------------------------------
\begin{itemize}
\item appraisal, technical planning, control and supervision of technical 
cooperation projects commissioned by the Government of the Federal Republic or 
by other authorities

\item providing an advisory service to other agencies also working on 
development projects

\item the recruitment, selection, briefing, assignment, administration of 
expert personnel and their welfare and technical backstopping during their 
period of assignment

\item provision of materials and equipment for projects, planning work, 
selection, purchasing and shipment to the developing countries

\item management of all financial obligations to the partner-country.
\end{itemize}
%-------------------------------------------------------------------------------
Deutsches Zentrum f\"{u}r Entwicklungstechnologien--GATE:

Deutsche Gesellschaft f\"{u}r Technische Zusammenarbeit (GTZ) GmbH

P.O. Box 5180

D-65726 Eschborn

Federal Republic of Germany


Tel.: (06196) 79-0

Telex: 41523-0 gtz d

Fax: (06196) 797352

