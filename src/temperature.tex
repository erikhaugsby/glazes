\chapter{Temperature Ranges and Requirements}
%-------------------------------------------------------------------------------
\section{What is Temperature?}
Temperature means the amount of heat energy in a material. We raise the 
temperature of a material by providing it with heat energy, using a fire or 
electricity. What effect does this have on a material? We know that many 
familiar substances can exist in different states of solid, liquid and gas. For 
example, water can exist as ice, liquid water or steam. What is different about 
it? Only the temperature. All materials consist of atoms and molecules which 
are in constant motion. The amount of motion depends on the temperature. Cold 
materials have less motion and therefore appear solid to us (e.g. ice). When 
the temperature is increased, the motion of the molecules becomes greater and 
they can move more freely around each other (e.g. water). When the temperature 
is increased even more, the molecules become very active, as we can see when 
water boils. Then the molecules are even less bonded together and we see gas 
(e.g. steam).

Similarly, glazes are solid when they are cold (at room temperature), liquid 
when they are heated sufficiently (in the kiln), and become gas when they are 
heated too much.

It is also important to understand the relationship between clay and glaze. 
Most common red clay (such as brick clay) melts by 1100\degree C. This makes it 
useful for forming low temperature products. 1200\degree C, it can be used as a 
glaze.
%-------------------------------------------------------------------------------
\section{Low-Temperature Range (900--1000\degree C)}
Products called earthenware, whiteware, low-temperature ceramics, and terra 
cotta are all fired in the range of 900--1100\degree C. We will call these 
products 
generally ``earthenware''. What they have in common are clay bodies that develop 
their maximum strength in this range, and glazes that are based on low-melting 
compounds such as lead, sodium and potassium.
%-------------------------------------------------------------------------------
\subsection{Advantages and Disadvantages}
%-------------------------------------------------------------------------------
\subsubsection{Advantages}
Low temperature ceramics have the advantage of easy firing--it is much simpler 
to construct kilns and burner systems that have to reach no more than 
1100\degree C, and fuel costs are lower. Bright colors are possible in this 
range. Most common clays cannot be fired higher than this.
%-------------------------------------------------------------------------------
\subsubsection{Disadvantages}
Earthenware is often not as strong as high temperature ware, because the clay 
does not become vitreous. This means that it also has some porosity (the 
property of absorbing water) with the result that earthenware products often do 
not hold water unless the glaze is perfectly fitted to the body. Also, it is 
easier to chip the glaze away from the clay.

Historically, many earthenware glazes were based on poisonous lead because it 
is easy to melt: nowadays this is not a problem because lead can be replaced by 
non-poisonous materials.

Modern earthenware glazes are usually based on frits, which are expensive--the 
lower firing cost must be compared to the higher cost of the glaze.
%-------------------------------------------------------------------------------
\subsection{Clay and Glaze Characteristics}
%-------------------------------------------------------------------------------
\subsubsection{Earthenware Clay}
Common red-burning clay is normally used, often mixed with talcum powder to 
increase its firing range. In many countries, red clay which contains lime is 
used because it makes it easier to formulate glazes that do not craze (crack). 
White firing clay bodies are often based on talc, ball clay and fluxes to make 
them harder.
%-------------------------------------------------------------------------------
\subsubsection{Earthenware Glaze}
Earthenware glazes are based on low-melting materials, mainly lead oxide (white 
lead oxide, red lead oxide), sodium and boron compounds (soda ash, borax, boric 
acid) and potassium compounds (pearl ash, also known as potassium carbonate). 
Usually it is necessary to use these compounds in the form of frits (see 
chapter on frits).
%-------------------------------------------------------------------------------
\subsection{Raw Material Requirements}
Most of the raw materials for low temperature glazes can be obtained from 
commonly available sources. They include: local clays, wood and rice husk ash, 
limestone, and even soap powder (based on sodium and boron compounds). 
Materials such as borax must be obtained from chemical suppliers. Ready-made 
frits can be obtained from glaze suppliers, but in many locations it is 
necessary to make them from raw materials.
%-------------------------------------------------------------------------------
\section{High-Temperature Range (1100--1300\degree C)}
Types of ware fired in this range are known as stoneware and porcelain.
%-------------------------------------------------------------------------------
\subsection{Advantages and Disadvantages}
%-------------------------------------------------------------------------------
\subsubsection{Advantages}
High temperature products are generally stronger, more  acid and 
abrasion-resistant. Raw materials do not require fritting. The clay is more 
vitreous and thus does not have problems of water seepage.
%-------------------------------------------------------------------------------
\subsubsection{Disadvantages}
Kilns for high temperatures require more sophisticated bricks and kiln 
furniture, and better burner systems. Fuel costs are higher.
%-------------------------------------------------------------------------------
\subsection{Appropriate Products}
High temperature products include stoneware utilitarian items, whiteware of 
various types, porcelain and electrical insulators.
%-------------------------------------------------------------------------------
\subsection{Clay and Glaze Characteristics}
%-------------------------------------------------------------------------------
\subsubsection{Stoneware Clay}
Clay body raw materials are limited to those clays which can withstand high 
temperatures without melting: fireclays, ball clays, china clays, "stoneware" 
clays. Most bodies also include feldspar to cause vitrification, which prevents 
water seepage through the body.
%-------------------------------------------------------------------------------
\subsubsection{Stoneware Glaze}
High temperature glaze is easier to make than the low temperature sort, mainly 
because it is not necessary to frit the ingredients.
%-------------------------------------------------------------------------------
\subsection{Raw Material Requirements}
Most stoneware and porcelain glazes are based on feldspar, quartz, limestone 
and clay, with other ingredients to provide specific properties of surface, 
color etc.
%-------------------------------------------------------------------------------
\section{Firing Systems and Glaze Effects}
Different types of kilns and fuels have specific effects on glaze color and 
surface.
%-------------------------------------------------------------------------------
\subsection{Oil, Gas, Wood, Coal, Electricity, Other}
These are the main options for fuel. Each fuel requires a different kiln design 
and burner system. You must first decide which fuel is most available and most 
economical. The choice of fuel will determine whether products can be 
open-fired on shelves, or whether it is necessary to use saggers to protect the 
glaze from ash and contamination from dirty fuel.

The cost of fuel should be thought about very carefully. One kg of fuel 
produces a certain amount of heat. Heat is usually measured in calories or in 
British Thermal Units (BTU). One calorie is the amount of heat required to 
raise the temperature of one cubic centimeter of water 1\degree C. The table at 
page 170 shows the heat value of different fuels. Because a calorie is very 
small, the usual unit of heat is expressed as kilocalories (kilo = 1000, so 1 
kilocalorie = 1000 calories).

A particular kiln, loaded with an average number of products and fired to a 
specific temperature, will usually require the same amount of fuel each time, 
since it requires a specific number of calories to convert raw clay and glaze 
into finished ceramics. When you know the total kg of products and the total 
cost of one firing, it is easy to calculate the cost per kg of product:

\(Total~Cost/Total~KG = Cost~per~KG\)

You can also calculate the total number of calories required to do one firing. 
If you are using kerosene, you can find from the table that one lifer of 
kerosene supplies about 12,000 kilocalories of heat. So, if you use 80 liters 
to do a firing, the calculation is:

\(Total~fuel*(kilocalories~per~unit) = total~kilocalories~required\)

\(80*12,000=960,000~kilocalories\)

When deciding on the type of fuel to use, you should find out the cost per 
kilocalorie for different fuels in your area.
%------------------------------------------------------------------------------
\subsubsection{Oil}
Oil is available in many different forms, all of which can be used by the 
potter, including kerosene, diesel, furnace oil, and waste crankcase oil. 
Kerosene is the most clean-burning (without too much smoke or impurities), and 
waste crankcase oil is the dirtiest to use. Normally, products can be 
open-fired, but oil will produce some discoloration. For high quality 
whiteware, saggars may be necessary. Oil is suitable for high or low 
temperatures.

Oil provides between 9000 and 11000 kilocalories per kg.
%------------------------------------------------------------------------------
\subsubsection{Gas}
Gas is available as natural gas, producer gas or liquid propane gas. Where gas 
is available at a reasonable cost (compared to other fuels), it is the easiest 
fuel to use. Gas is very clean-burning, does not require saggars, and the 
burners are also simple to manufacture locally. It is suitable for any 
temperature.
%------------------------------------------------------------------------------
\subsubsection{Wood}
Almost any kind of wood can be used for firing kilns. Nowadays, wood is a 
scarce resource in most countries and more and more it is being replaced by 
other fuels. Firing with wood is labor-intensive. Because it produces a large 
volume of ash, it is usually necessary to fire the ware in saggers. It is 
suitable for any temperature.

On the other hand, wood is a renewable resource and in many areas of the world 
it is produced as a cash crop, which makes it appropriate to use.

The calorific value of wood is difficult to calculate, because it depends on 
the type of wood, whether it is wet or dry, and the efficiency of burning. Dry 
wood can supply between 3000 and 4500 kilocalories per kg, whereas the same 
wood when wet may produce only half the calories.
%------------------------------------------------------------------------------
\subsubsection{Coal}
Coal comes in many different grades, all of which are suitable for firing 
kilns. Firing with coal is labor-intensive, but in many countries it is the 
cheapest fuel available. Coal also produces ash and impurities, so it is 
usually necessary to fire the ware in saggars. It is best for high 
temperatures, but can be used at any temperature.

Coal can provide between 4500 and 7700 kilocalories per kg.
%------------------------------------------------------------------------------
\subsubsection{Electricity}
Electric kilns are practical for the small producer where there is a reliable 
source of electricity. Because there is no combustion, electricity is the 
cleanest fuel of all. Electric kilns fire very evenly and do not require 
saggers. Electricity is best for temperatures up to 1100\degree C.
%------------------------------------------------------------------------------
\subsubsection{Other Fuels}
These include tires, which burn very well but produce a lot of smoke, and also 
produce poisonous gases. They can be used in kilns designed to burn wood or 
coal. Some brick industries use scrap asphalt from roads as fuel. Also in this 
category are such fuels as brushwood, sawdust and rice husk. Most of these are 
dirty-burning, so require the use of saggers. They are best for low 
temperatures.
%------------------------------------------------------------------------------
\subsection{Oxidation and Reduction}
To understand oxidation and reduction, it is necessary to know how fuel burns. 
All fuel produces heat when it combines with the oxygen in the air. As anyone 
knows who has made a wood fire, if there is plenty of air the fire burns hot 
and clean, with little smoke. This is called an oxidation fire. If the air is 
reduced, there will be less heat and more smoke. This is called a reduction (or 
reducing) fire, which simply means reducing the amount of oxygen. So:
%------------------------------------------------------------------------------
\begin{itemize}
\item Oxidation firing means there is plenty of air and no smoke.
\item Reduction firing means there is little air and more smoke.
\end{itemize}
%------------------------------------------------------------------------------
Glazes will have different colors and surfaces depending on whether they are 
fired in oxidation or reduction conditions. Oxidation has its greatest effect 
on the metallic oxides that are used to create color in glazes. 

For example:
%------------------------------------------------------------------------------
\begin{center}
        \renewcommand{\arraystretch}{1.5}
        \begin{tabular}{|c|c|c|}\hline
\textbf{Oxide}&\textbf{Oxidation}&\textbf{Reduction}\\\hline\hline
%------------------------------------------------------------------------------
Red Iron Oxide&Brown&Red-Brown, Black\\\hline
%------------------------------------------------------------------------------
Copper Oxide&Green, Blue&Red\\\hline
%------------------------------------------------------------------------------
\end{tabular}
\end{center}
%------------------------------------------------------------------------------
Iron also changes from a grey color to a red color when it rusts. This is 
because oxide from the air-combines with the metal and forms iron oxide.

In firing, it is difficult to exactly control the amount of oxidation or 
reduction. Many beautiful glazes can be obtained in reduction firing, so it is 
widely used for decorative stoneware, and for lusterware. However, the results 
are variable and difficult to reproduce every time, and even in one kiln-load 
there will be differences. For that reason, most producers who need to supply a 
uniform product use oxidation firing.
%------------------------------------------------------------------------------
\subsection{Vapor Glazing}
In vapor glazing techniques, the glaze is not applied to the product before 
firing in the usual manner. Instead, glaze is introduced into the kiln through 
the firebox at the end of the firing, when there is enough heat to change the 
glaze into vapor form. The most common material for vapor glazing is ordinary 
salt. At temperatures above 1100\degree C, salt breaks down into sodium and 
chlorine vapor, which circulates through the kiln. The sodium is attracted to 
silica in the clay and forms a strong, durable glaze. Salt glazing is used 
mainly for sewage pipes, because it is cheap and a perfectly glazed surface is 
not necessary. In Europe, it was once used widely for household items, even 
including beer bottles. Nowadays, salt glazing is less popular because it 
produces toxic smoke that harms the environment.

Salt is sometimes replaced by soda ash and sodium bicarbonate, which produce a 
similar vapor glaze without the poisonous side effects. Vapor glazing is not 
recommended for the small producer, except for making specialized art ceramics.
%------------------------------------------------------------------------------