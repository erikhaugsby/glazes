\chapter{Glaze Formula Calculations}
Glazes are expressed in several different forms:
%-------------------------------------------------------------------------------
\begin{itemize}
\item Recipe: a list of actual materials and weights, used directly to make the 
glaze.

\item Molecular formula: shows the relative proportion of molecules of flux, 
alumina and silica in the glaze. Must be converted to recipe to make the glaze.

\item Chemical analysis: shows the percentage of oxides in the glaze. Also 
known as ultimate composition.

\item Seger formula: a special molecular formula, which makes it easier to 
compare glazes. It is also known as the ``empirical formula''.
\end{itemize}
%-------------------------------------------------------------------------------
\section{Glaze Formula Chemistry}
Why glaze formulas?

As you already know, glaze materials are complicated and if you only work with 
clay, limestone, talc, quartz etc. there is no way to theoretically understand 
how they combine in the glaze. For this reason, in order to make glazing 
scientific and systematic, it is necessary to use chemistry. This makes it 
possible to write materials as chemical symbols and to make calculations that 
help to invent new glazes and to alter existing recipes.
%-------------------------------------------------------------------------------
\subsection{Using Chemical Symbols}
Chemical symbols are a language for describing atoms, molecules and the way 
they are combined to make up the various materials used in chemistry and in 
glazes.

As already described at the beginning of the book, there are more than 100 
elements, which are the basic building blocks of glaze materials. Each one has 
a chemical symbol:

Calcium = Ca

Copper = Cu

Iron = Fe

etc.
%-------------------------------------------------------------------------------
\subsection{Chemical Reactions}
Elements are usually not found by themselves in nature. The basic nature of 
elements is to combine with each other: this process is called a chemical 
reaction and takes place in nature through the effects of heat, pressure etc. 
When elements combine, they are called compounds and they can be described by 
chemical formulas, which show the number of atoms and how they are attached to 
each other.

For example, china clay is written as \ce{Al2O3*2SiO2*2H2O}. Each element is 
followed by a number written below the line: this is the number of atoms in the 
compound.

\ce{Al2} means 2 atoms of alumina and \ce{O3} means 3 atoms of oxygen. This is 
the compound aluminum oxide.

If no number follows the element symbol, it is understood to be only 1 atom.

The raised period (·) shows that the compounds are joined together chemically 
to form a complex compound. The large numbers before each compound mean the 
number of molecules that combine. If there is no number in front, it is 
understood to mean 1 molecule.

\ce{Al2O3} means 1 molecule of aluminum oxide. \ce{2SiO2} means 2 molecules of 
silicon oxide.

So \ce{AL2O3*2SiO2*2H2O} is a complex compound consisting of 1 molecule of 
aluminum oxide, 2 molecules of silicon oxide and 2 molecules of water.

These compounds cannot be broken down physically but can combine with other 
compounds when heated sufficiently in the kiln.
%-------------------------------------------------------------------------------
\subsection{Molecular Weights}
Each kind of molecule has a specific weight. We all know that 1 kg of lead is 
much smaller than 1 kg of aluminum. This is because the molecules are heavier 
and are packed together more closely.

Because it is impossible to weigh individual molecules, they have all been 
assigned molecular weights, which are relative to hydrogen, which has been 
given the molecular weight of 1. The molecular weights of all the other 
elements are based on how much heavier they are compared to hydrogen.

So the molecular weight of oxygen = 16, meaning it is 16 times heavier than 
hydrogen.
%-------------------------------------------------------------------------------
\subsection{Formula Weight of Minerals}
The molecular weights of all the elements in a compound can be added together 
to get the total molecular weight. This is called the formula weight. In our 
example of kaolin clay, we can look in the table of elements and oxides in the 
appendix to find out the individual molecular weights. Molecular weight is 
abbreviated to ``MW''. In order to simplify calculations we round up the MW 
figures. This is accurate enough since we seldom know the exact composition of 
our raw materials anyway.

This is known as formula weight.

As an example, table~\ref{tab:formulaweightkaolin} shows the calculation of 
the formula weight of kaolin as 258.
%-------------------------------------------------------------------------------
\begin{landscape}
\begin{center}
    \renewcommand{\arraystretch}{1.5}
\begin{table}
  \centering
  \begin{tabular}{|c|c|c|c|c|c|}\hline
\textbf{Element}&\textbf{Molecular Weight}&\textbf{Number of 
Atoms}&&\textbf{Oxide Weight}&\textbf{Compound Weight}\\\hline\hline
AL&27&2&$2*27=54$&&\\\hline
O&16&3&$3*16=48$&102&$1*102=102$\\\hline
Si&28&1&$1*28=28$&&\\\hline
O&16&2&$2*16=32$&60&$2*60=120$\\\hline
H&1&2&$2*1=2$&&\\\hline
O&16&1&$1*16=16$&18&$2*18=36$\\\hline\hline
&&&&\textbf{Total compound weight =}&\textbf{258}\\\hline
  \end{tabular}
\caption{The formula weight of kaolin calculated as 258.}
\label{tab:formulaweightkaolin}
\end{table}
\end{center}
\end{landscape}
%-------------------------------------------------------------------------------
\subsection{Percentage to Formula}
Glaze formulas are often given as percentages of the various oxides. In order 
to find out the chemical formula, the rule is to divide each oxide by its 
molecular weight.

In the appendix you will find the molecular weight of glaze oxide and materials.

The table~\ref{tab:molecularformulakaolin} shows a calculation of the molecular 
formula of kaolin. As shown in the table, the molecular formula of kaolin is:

\ce{0.387Al2O3*0.775SiO2*0.775H2O}.
%-------------------------------------------------------------------------------
\begin{center}
  \renewcommand{\arraystretch}{1.5}
\begin{table}\centering
  \begin{tabular}{|c|c|c|c|c|}\hline
    \textbf{Oxide}&\textbf{Symbol}&\textbf{Percent}&\textbf{Molecular 
    Weight}&\textbf{Calculation}\\\hline\hline
    Silica&\ce{SiO2}&46.51\%&60&$46.51/60=0.775$\\\hline
    Alumina&\ce{Al2O3}&39.53\%&102&$39.52/102=0.387$\\\hline    
    Water&\ce{H2O}&13.96\%&18&$13.96/18=0.775$\\\hline
      \end{tabular}
\caption{The molecular formula of kaolin.}
\label{tab:molecularformulakaolin}
\end{table}
\end{center}
%-------------------------------------------------------------------------------
Because this is difficult to use, we divide all the numbers by the smallest one.

$0.387/0.387 = \ce{1Al2O3}$

$0.775/0.387 = \ce{2SiO2}$

$0 775/0.387 = \ce{2H2O}$

The formula comes out neatly as the familiar \ce{Al2O3*2SiO2*2H2O}, or kaolin.

For using a material in glaze calculation we need to calculate its formula 
weight. This is done for kaolin as shown in table~\ref{tab:formulaweightkaolin} 
above.
%-------------------------------------------------------------------------------
\section{Seger Formula}
About 100 years ago a German ceramist, Hermann Seger, developed Seger cones for 
measuring temperatures in kilns. He also proposed writing the composition of 
glazes according to the number of different oxides in the glaze instead of 
listing the raw materials used in the glaze.

For example: Aluminum oxide can be added to the glaze either in the form of 
clay (\ce{Al2O3*2SiO2*2H2O}) or feldspar (\ce{K2O*Al2O3*6SiO2}).

Seger formulas allow all glaze formulas to be expressed in a table, keeping the 
groups separate in order to make comparison of different formulas easy (see 
below).

The organization of the Seger formula is always according to 
table~\ref{tab:segerformulaorg}.

In the table form, the sum of the fluxes must always equal 1, which makes 
different formulas easy to compare.

The oxides used in glazes are divided into three groups according to the way 
the oxides work in the glaze.

\textbf{Note:} \ce{B2O3} is sometimes listed under stabilizers and sometimes 
under glass formers, since it has both characteristics.
%-------------------------------------------------------------------------------
\subsubsection{Fluxes}
This group of oxides functions as melter, and fluxes are also called basic 
oxides or bases. They are written \ce{RO} or \ce{R2O}, where R represents any 
atom and O represents oxygen. So all the fluxes are a combination of one or two 
element atoms and one oxygen atom.
%-------------------------------------------------------------------------------
\subsubsection{Stabilizers}
These work as stiffeners in the melted glaze to prevent it from running too 
much. They are considered neutral oxides and are writen as \ce{R2O3} or two 
atoms of some element combined with three oxygen atoms.
%-------------------------------------------------------------------------------
\subsubsection{Glass formers}
These form the noncrystalline structure of the glaze. They are called acidic 
oxides and are written as \ce{RO2} or one element atom combined with two oxygen 
atoms.
%-------------------------------------------------------------------------------
\begin{center}
  \renewcommand{\arraystretch}{1.5}
  \begin{table}\centering
    \begin{tabular}{|c|c|c|}\hline
      \textbf{Fluxes}&\textbf{Stabilizer}&\textbf{Glass Formers}\\\hline\hline
      \textbf{\ce{RO}}, 
      \textbf{\ce{R2O}}&\textbf{\ce{R2O3}}&\textbf{\ce{RO2}}\\\hline\hline
      \textbf{Alkalis:}&\ce{Al2O3}&\ce{SiO2}\\\hline
      \ce{K2O}&\ce{B2O3}&\ce{TiO2}\\\hline
      \ce{Na2O}&\ce{B2O3}&\\\hline
      \ce{Li2O}&&\\\hline\hline
      \textbf{Alkaline earths:}&&\\\hline
      \ce{CaO}&&\\\hline
      \ce{MgO}&&\\\hline
      \ce{BaO}&&\\\hline\hline
      \textbf{Others:}&&\\\hline
      \ce{PbO}&&\\\hline
      \ce{ZnO}&&\\\hline
    \end{tabular}
    \caption{Organization of the Seger Formula.}
    \label{tab:segerformulaorg}
  \end{table}
\end{center}
%-------------------------------------------------------------------------------
\subsection{Table of Limit Formulas}
Tables \ref{tab:glazelimitsearth} and \ref{tab:glazelimitsstone} shows limits 
for glaze ingredients in earthenware and stoneware glazes, as provided by 
Daniel Rhodes in \textit{Clay and Glazes for the Potter}.

\textbf{Note:} \ce{KNaO} is a symbol for either sodium oxide or potassium oxide.
%-------------------------------------------------------------------------------
\begin{center}
  \renewcommand{\arraystretch}{1.5}
  \begin{table}\centering
    \begin{tabular}{|c|c|c|c|c|c|}\hline
    \multicolumn{6}{|c|}{\textbf{c012--08: Lead Glazes}}\\\hline\hline
    \textbf{Oxide}&\textbf{Limit}&\textbf{Oxide}&\textbf{Limit}&\textbf{Oxide}&\textbf{Limit}\\\hline
    \ce{PbO}&0.7--1.0&\ce{Al2O3}&0.05--0.2&\ce{SiO2}&1.0--1.5\\\hline
    \ce{KNaO}&0--0.3&&&&\\\hline
    \ce{ZnO}&0--0.1&&&&\\\hline
    \ce{CaO}&0--0.2&&&&\\\hline\hline
    \multicolumn{6}{|c|}{\textbf{c08--01: Lead Glazes}}\\\hline\hline
\textbf{Oxide}&\textbf{Limit}&\textbf{Oxide}&\textbf{Limit}&\textbf{Oxide}&\textbf{Limit}\\\hline
\ce{PbO}&0.7--1.0&\ce{Al2O3}&0.1--0.25&\ce{SiO2}&1.5--2.0\\\hline
\ce{KNaO}&0--0.3&&&&\\\hline
\ce{ZnO}&0--0.2&&&&\\\hline
\ce{CaO}&0--0.3&&&&\\\hline\hline
    \multicolumn{6}{|c|}{\textbf{c08--04: Alkaline Glazes}}\\\hline\hline
    \textbf{Oxide}&\textbf{Limit}&\textbf{Oxide}&\textbf{Limit}&\textbf{Oxide}&\textbf{Limit}\\\hline
    \ce{PbO}&0--0.5&\ce{Al2O3}&0.5--0.25&\ce{SiO2}&1.5--2.5\\\hline
\ce{KNaO}&0.4--0.8&&&&\\\hline
\ce{ZnO}&0--0.2&&&&\\\hline
\ce{CaO}&0--0.3&&&&\\\hline\hline
    \multicolumn{6}{|c|}{\textbf{c08--04: Lead-Boron Glazes}}\\\hline\hline
    \textbf{Oxide}&\textbf{Limit}&\textbf{Oxide}&\textbf{Limit}&\textbf{Oxide}&\textbf{Limit}\\\hline
    \ce{PbO}&0--0.5&\ce{Al2O3}&0.15--0.2&\ce{SiO2}&1.5--2.5\\\hline
\ce{KNaO}&0.1--0.25&&&\ce{B2O3}&0.15--0.6\\\hline
\ce{ZnO}&0.1--0.25&&&&\\\hline
\ce{CaO}&0.3--0.6&&&&\\\hline
\ce{BaO}&0--0.15&&&&\\\hline
    \end{tabular}
\caption{Limits for earthenware glaze ingredients.}
\label{tab:glazelimitsearth}
\end{table}
\end{center}
%-------------------------------------------------------------------------------
\begin{center}
  \renewcommand{\arraystretch}{1.5}
  \begin{table}\centering
    \begin{tabular}{|c|c|c|c|c|c|}\hline
    \multicolumn{6}{|c|}{\textbf{c2--5: Lead Glazes}}\\\hline\hline    
    \textbf{Oxide}&\textbf{Limit}&\textbf{Oxide}&\textbf{Limit}&\textbf{Oxide}&\textbf{Limit}\\\hline
    \ce{PbO}&0.4--0.6&\ce{Al2O3}&0.2--0.28&\ce{SiO2}&2.0-3.0\\\hline
\ce{KNaO}&0.1--0.25&&&\ce{B2O3}&0.15--0.6\\\hline
\ce{ZnO}&0.1--0.25&&&&\\\hline
\ce{CaO}&0.1--0.4&&&&\\\hline\hline
    \multicolumn{6}{|c|}{\textbf{c2--5: Boron}}\\\hline\hline    
    \textbf{Oxide}&\textbf{Limit}&\textbf{Oxide}&\textbf{Limit}&\textbf{Oxide}&\textbf{Limit}\\\hline
    \ce{KNaO}&0.1--0.25&\ce{Al2O3}&0.2--0.28&\ce{SiO2}&2.0--3.0\\\hline
\ce{ZnO}&0.1--0.25&&&\ce{B2O3}&0.15--0.6\\\hline
\ce{CaO}&0.2--0.5&&&&\\\hline
\ce{BaO}&0.1--0.25&&&&\\\hline\hline
    \multicolumn{6}{|c|}{\textbf{c8--12: Stoneware Glazes}}\\\hline\hline    
    \textbf{Oxide}&\textbf{Limit}&\textbf{Oxide}&\textbf{Limit}&\textbf{Oxide}&\textbf{Limit}\\\hline
    \ce{KNaO}&0.2--0.4&\ce{Al2O3}&0.3--0.5&\ce{SiO2}&3.0-5.0\\\hline
\ce{ZnO}&0--0.3&&&{B2O3}&0.2--0.6\\\hline
\ce{CaO}&0.4--0.7&&&&\\\hline
\ce{BaO}&0--0.3&&&&\\\hline
\ce{MgO}&0--0.3&&&&\\\hline
    \end{tabular}
\caption{Limits for stoneware glaze ingredients.}
\label{tab:glazelimitsstone}
\end{table}
\end{center}
%-------------------------------------------------------------------------------
For example, a simple unfritted lead glaze would look as shown in 
table~\ref{tab:calculationsunfrittedleadrecipe}.
%-------------------------------------------------------------------------------
\begin{center}
\renewcommand{\arraystretch}{1.5}
\begin{table}\centering
\begin{tabular}{|c|c|c|c|c|c|}\hline
 \multicolumn{2}{|c|}{\textbf{Fluxes}}
&\multicolumn{2}{|c|}{\textbf{Stabilizer}}
&\multicolumn{2}{|c|}{\textbf{Glass former}}\\\hline\hline
 \multicolumn{2}{|c|}{RO, \ce{R2O}}
&\multicolumn{2}{|c|}{\ce{R2O3}}
&\multicolumn{2}{|c|}{\ce{RO2}}\\\hline
PbO&1&\ce{Al2O3}&0.1&\ce{SiO2}&1.5\\\hline
\end{tabular}
\caption{The Seger formula for a simple unfritted lead glaze. Remember that the 
flux column always totals 1.0.}
\label{tab:calculationsunfrittedleadrecipe}
\end{table}
\end{center}
%-------------------------------------------------------------------------------
A more complicated formula is the unfritted boron glaze is shown in 
table~\ref{tab:calculationsunfrittedboronformula}.
%-------------------------------------------------------------------------------
\begin{center}
  \renewcommand{\arraystretch}{1.5}
  \begin{table}\centering
    \begin{tabular}{|c|c|c|c|c|c|}\hline
      \multicolumn{2}{|c|}{\textbf{Fluxes}}
      &\multicolumn{2}{|c|}{\textbf{Stabilizer}}
      &\multicolumn{2}{|c|}{\textbf{Glass Former}}\\\hline\hline
      CaO&0.414&\ce{Al2O3}&0.322&\ce{SiO2}&2.291\\\hline
      MgO&0.414&&&\ce{B2O3}&0.931\\\hline
      \ce{K2O}&0.172&&&&\\\hline\hline
      &1.000&&&&\\\hline
    \end{tabular}
    \caption{The Seger formula for an unfritted boron glaze.}
    \label{tab:calculationsunfrittedboronformula}
  \end{table}
\end{center}
%-------------------------------------------------------------------------------
There are some basic rules for the ratio of oxides in the 3 different groups, 
according to glaze temperature. These are called limit formulas. They should 
only be considered guidelines, as many glazes exceed the limits in practice.
%-------------------------------------------------------------------------------
\begin{itemize}
\item Addition of 0.1 parts \ce{SiO2} to a glaze will increase the melting 
point by about 20\degree C.
\item Addition of 0.05 parts \ce{B2O3} will lower the melting point by 
20\degree C.
\end{itemize}
%-------------------------------------------------------------------------------
The formulas of pyrometric Seger cones are listed in the appendix. These can 
also be used as a guide for glazes by choosing a cone formula 4 to 5 cones 
below the glaze firing temperature. If you need a glaze for cone 9, 1280\degree 
C, you can use the cone 5 formula for the glaze.
%-------------------------------------------------------------------------------
\subsection{Benefits of Using the Seger Formula}
The main usefulness of the Seger formula is that it presents glazes in a way 
that is easy to compare. 

The Seger formula should be considered a guide only, as most theoretical glazes 
do not react as expected and still require empirical testing to develop them 
fully. If you want to use Seger formulas for your glazes it is nice to have 
exact chemical analysis of your raw materials, but this is seldom the case. 
Instead you will have to pick one of the materials listed in the appendix. They 
may be close enough for practical work.
%-------------------------------------------------------------------------------
\subsubsection{Originating new glazes}
Glazes with desired characteristics of color, mattress etc. can first be 
written as Seger formulas, selecting oxides that are known to produce the 
effects.
%-------------------------------------------------------------------------------
\subsubsection{Comparing glaze recipes}
It is difficult to look at two recipes and see how they are different. If they 
are converted into Seger formulas, the differences can easily be seen.
%-------------------------------------------------------------------------------
\subsubsection{Substituting materials}
If a material is no longer available, other materials can be substituted by 
working out the quantities in the Seger formula.
%-------------------------------------------------------------------------------
\subsubsection{Modifying glazes}
Glazes that change character, have problems etc. can be analyzed as Seger 
formulas, and directions for testing decided.
%-------------------------------------------------------------------------------
\subsection{Glaze Recipe from Formula}
To get the glaze recipe from the formula, there is a standard series of 
calculations.
\subsubsection{Example: finding the material percentages in a lead glaze}
This example uses the simple lead glaze as shown in 
table~\ref{tab:calculationsunfrittedleadrecipe}.

First, decide which raw materials to use. For lead oxide, PbO, the choices are 
red lead, white lead or litharge. \ce{Al2O3} is almost always obtained from 
china clay, and \ce{SiO2} usually from quartz powder.

The calculation is helped with a table like 
table~\ref{tab:glazerecipefromformula}.
%-------------------------------------------------------------------------------
\begin{center}
  \renewcommand{\arraystretch}{1.5}
  \begin{table}\centering
    \begin{tabular}{|c|c|c|c|c|}\hline
      \textbf{Material/Formula}&\textbf{Mols}&\textbf{Pb=1.0}&\textbf{\ce{Al2O3}=0.1}
      &\textbf{\ce{SiO2}=1.5}\\\hline\hline
      Litharge, PbO&1.0&1.0&-&-\\\hline
      Kaolin, \ce{Al2O3*2SiO2*2H2O}&0.1&-&0.1&0.2\\\hline
      Quartz, \ce{SiO2}&1.3&-&-&1.3\\\hline\hline
      \textbf{Total}&\textbf{-}&\textbf{1.0}&\textbf{0.1}&\textbf{1.5}\\\hline
    \end{tabular}
    \caption{Finding the required molecular parts of each material in the 
    simple lead glaze recipe.}
    \label{tab:glazerecipefromformula}
  \end{table}
\end{center}
%-------------------------------------------------------------------------------
1.0 molecular parts of litharge provides all the PbO needed. We enter kaolin 
and its formula in the table and write 0.1 for MP. When we take 0.1 part 
kaolin, we get 0.1 \ce{Al2O3} and we enter this on the right. In the kaolin 
formula we have \ce{2SiO2} so when we take 0.1 MP of kaolin we get 0.2 
\ce{SiO2}. We list this under \ce{SiO2}. We need 1.5 \ce{SiO2}, so 1.3 remains 
and we get this from quartz.

Next the required molecular parts of each material are multiplied by their 
molecular weights to get the batch weight of each material.
%-------------------------------------------------------------------------------
\begin{center}
  \renewcommand{\arraystretch}{1.5}
  \begin{table}\centering
    \begin{tabular}{|c|c|c|c|c|}\hline
      \textbf{Material}&\textbf{MP}&\textbf{MW}&\textbf{Calculation}
      &\textbf{Batch Weight}\\\hline\hline
      Litharge&1.0&223&$1*223$&223\\\hline
      Kaolin&0.1&258&$0.1*258$&25.8\\\hline
      Quartz&1.3&60&$1.3*60$&70.8\\\hline\hline
      &&&\textbf{Total}&326.8\\\hline
    \end{tabular}
    \caption{The batch weight of the glaze shown in 
    table~\ref{tab:glazerecipefromformula}.}
    \label{tab:glazerecipefromformulaweights}
  \end{table}
\end{center}
%-------------------------------------------------------------------------------
To change this recipe into percentages, all the figures are divided by the 
total, as shown in table~\ref{tab:glazerecipefromformulatotal}.
%-------------------------------------------------------------------------------
\begin{center}
  \renewcommand{\arraystretch}{1.5}
  \begin{table}\centering
    \begin{tabular}{|c|c|c|c|}\hline
      \textbf{Material}&\textbf{Calculation}&\textbf{Decimal}&\textbf{Percentage}
\\\hline\hline
      Litharge&$223/326.8=$&0.68&68\%\\\hline
      Kaolin&$25.8/326.8=$&0.8&8\%\\\hline
      Quartz&$70.8/326.8=$&0.24&24\%\\\hline\hline
      &&\textbf{Total}&100\%\\\hline
    \end{tabular}
    \caption{Converting molecular weights from      
    table~\ref{tab:glazerecipefromformulaweights} into percentages.}
    \label{tab:glazerecipefromformulatotal}
  \end{table}
\end{center}
%-------------------------------------------------------------------------------
\subsubsection{Example: finding the material percentages in a boron glaze}
%-------------------------------------------------------------------------------
A more complicated formula is the unfritted boron glaze, as shown in 
table~\ref{tab:calculationsunfrittedboronformula}.

Again, the first step is to select materials. Because materials that supply 
more than one oxide usually work better in glazes, they are preferred if 
available. We need both CaO and MgO, which are supplied by dolomite, 
\ce{CaCO3*MgCO3}. Potash feldspar supplies \ce{K2O} along with \ce{Al2O3} and 
\ce{SiO2}. Quartz provides \ce{SiO2}. For boron, boric acid is selected.
%-------------------------------------------------------------------------------
The calculation procedure is as follows:
\begin{enumerate}
\item Enter formula at top of calculation table.
\item Select materials, enter formula and MW.
\item Multiply each material's MW with its MW and enter result in part's weight.
\item Enter MP of each oxide of the material under the formula to check oxide 
balance.
\item Convert parts' weight into a percentage recipe.
\end{enumerate}
As before we change the recipe to percentage. When calculating from formula to 
recipe, there is no need to carry out results beyond round figures, 
particularly when we do not know the exact chemical analysis of our materials.
%-------------------------------------------------------------------------------
%-------------------------------------------------------------------------------
\begin{center}
  \renewcommand{\arraystretch}{1.5}
  \begin{table}\centering
    \begin{tabular}{|c|c|c|c|}\hline
      \textbf{Material}&\textbf{Calculation}&\textbf{Decimal}&\textbf{Percentage}
      \\\hline\hline
      Dolomite&$384/76=$&19.8&20\%\\\hline
      Potash feldspar&$384/96=$&25&25\%\\\hline
      Kaolin&$384/39=$&10.2&10\%\\\hline
      Quartz&$384/58=$&15.1&15\%\\\hline\hline
      Boric acid&$384/115=$&29.9&30\%\\\hline
      &&\textbf{Total}&100\%\\\hline
    \end{tabular}
    \caption{Converting molecular weights from      
      table~\ref{tab:calculationsunfrittedboronformula} into percentages.}
    \label{tab:glazerecipefromformulatotal}
  \end{table}
\end{center}
%-------------------------------------------------------------------------------
Calculating from a recipe to the Seger formula is the same process in reverse. 
We will use the same raw boric acid glaze as an example. Again we use the 
calculation table~\ref{tab:calculationsunfrittedboronformula} and the following 
steps.
%-------------------------------------------------------------------------------
\begin{enumerate}
\item Enter recipe materials and their formulas in the left column and MW and 
recipe figures in MP's weight column.
\item Write oxides of the materials at top of table.
\item Divide each recipe figure with its MW and enter result under MP.
\item Multiply MP with each oxide in material formula and enter result under 
respective oxide in the right columns.
\item Add together all oxides and list them according to RO-\ce{R2O3}-\ce{RO2}.
\item Add oxides in RO and divide all RO figures with the total.
\end{enumerate}
%-------------------------------------------------------------------------------
Note that from dolomite only CaO and MgO are entered in the formula. \ce{CO2} 
is released during heating and does not take part in the glaze melt. \ce{H2O} 
of kaolin and boric acid likewise evaporates.

The oxides are set up in the standard Seger formula. Then the formula is 
brought to unity by dividing all the figures by the total in the left column, 
as shown in table~\ref{tab:calculationsunfrittedboronformulaunity}.

\textbf{Note:} The figures are not exactly the same as the original 
formula above, due to rounding off the figures. This is accurate enough for 
practical work.

If you have a chemical analysis of materials you want to use in a glaze, you 
first have to calculate the formula of the material. Then you enter this 
formula and its formula weight in the table under MW.
%-------------------------------------------------------------------------------
\begin{center}
  \renewcommand{\arraystretch}{1.5}
  \begin{table}\centering
    \begin{tabular}{|c|c|c|c|c|c|}\hline
      \multicolumn{2}{|c|}{\textbf{Fluxes}}
      &\multicolumn{2}{|c|}{\textbf{Stabilizer}}
      &\multicolumn{2}{|c|}{\textbf{Glass Former}}\\\hline\hline
      CaO&0.109&\ce{Al2O3}&0.84&\ce{SiO2}&0.598\\\hline
      MgO&0.109&&&\ce{B2O3}&0.244\\\hline
      \ce{K2O}&0.045&&&&\\\hline\hline
      &0.263&&&&\\\hline
    \end{tabular}
    \caption{The Seger formula for the unfritted boron glaze from 
    table~\ref{tab:calculationsunfrittedboronformula}.}
    \label{tab:calculationsunfrittedboronformula2}
  \end{table}
\end{center}
%-------------------------------------------------------------------------------
\begin{center}
  \renewcommand{\arraystretch}{1.5}
  \begin{table}\centering
    \begin{tabular}{|c|c|c|c|c|c|}\hline
      \multicolumn{2}{|c|}{\textbf{Fluxes}}
      &\multicolumn{2}{|c|}{\textbf{Stabilizer}}
      &\multicolumn{2}{|c|}{\textbf{Glass Former}}\\\hline\hline
      CaO&0.414&\ce{Al2O3}&0.322&\ce{SiO2}&2.291\\\hline
      MgO&0.414&&&\ce{B2O3}&0.931\\\hline
      \ce{K2O}&0.172&&&&\\\hline\hline
    \end{tabular}
    \caption{The unified Seger formula for an unfritted boron glaze from 
    table~\ref{tab:calculationsunfrittedboronformula2}.}
    \label{tab:calculationsunfrittedboronformulaunity}
  \end{table}
\end{center}
%-------------------------------------------------------------------------------
\section{Frit Calculation}
Frit calculation is done in the same way as calculating a glaze, but the 
calculation is slightly more complicated. As with glazes it is important to 
follow the recipes accurately.

This also means that you have to make sure that the raw materials are not wet 
when you weigh them. Also remember that materials like calcined soda and borax 
will absorb moisture from the air if they are not kept in a sealed container.
%-------------------------------------------------------------------------------
\subsection{Moisture Compensation}
If you have to weigh materials with a high moisture content you can compensate 
for this. Weigh 100 g of the material, dry it and then weigh it again. 

Moisture content is: $((wet~weight-dry~weight)*100)/(dry~weight)=x\%$

This $x\%$ is added to the amount you are weighing to compensate for its 
moisture content. 

For example: 100 g of kaolin weights 92 g after drying. This can be caluclated 
as follows: $(100-92)*100)/92=8.7\%$.

Table~\ref{tab:moisturecompensation} shows how to find the total amount of 
kaolin actually needed for a frit, when its moisture content is 8.7%.
%-------------------------------------------------------------------------------
\begin{center}
  \renewcommand{\arraystretch}{1.5}
  \begin{table}[htbp!]\centering
    \begin{tabular}{|c|c|c|}\hline
      \multicolumn{3}{|c|}{\textbf{Kaolin moisture content: 
      $(100-92)*100)/92=8.7\%$}}\\\hline\hline
      Kaolin in recipe&&3500 g\\\hline
      Compensation&$8.7\%*3500$&304.5 g\\\hline
      Total amount of kaolin needed&$3500 + 304.5$g&3804.5 g\\\hline
    \end{tabular}
    \caption{Finding moisture content of kaolin, and compensating.}
    \label{tab:moisturecompensation}
  \end{table}
\end{center}
%-------------------------------------------------------------------------------
\subsection{Formula Rules for Frit}
The practice of fritting was described in chapter~\ref{sec:frits}. The main 
reason for fritting is to make glaze materials insoluble, which is possible if 
the frit materials are mixed in the right proportion. In formula terms they 
should fall within these limits:
%-------------------------------------------------------------------------------
\begin{itemize}
\item Ratio flux: \ce{SiO2} should be between $1:1.5$ and $1:3$
\item The sum of \ce{K2O} and \ce{Na2O} should not exceed 0.5 molecular parts 
on the flux side. The other 0.5 can be filled by fluxes like PbO, CaO, ZnO, and 
BaO.
\item The ratio of $\ce{B2O3}:\ce{SiO2}$ should not be less than $1:2$, but 
with other materials like PbO, CaO, MgO, or \ce{K2O} in the formula this 
proportion can go down to $1:1.5$
\item At least 0.05 molecular parts of \ce{Al2O3} reduces solubility, but 
it should not exceed 0.2 molecular parts because this reduces the fluidity of 
the frit melt.
\end{itemize}
%-------------------------------------------------------------------------------
\subsection{Frit Based on Glaze Formula}
Imagine a glaze formula of an opaque boron glaze to melt at 1100\degree C, as 
shown in table~\ref{tab:boronglazefrit}
%-------------------------------------------------------------------------------
\begin{center}
  \renewcommand{\arraystretch}{1.5}
  \begin{table}\centering
    \begin{tabular}{|c|c|c|c|c|c|}\hline
      \multicolumn{2}{|c|}{\textbf{Fluxes}}
      &\multicolumn{2}{|c|}{\textbf{Stabilizer}}
      &\multicolumn{2}{|c|}{\textbf{Glass Former}}\\\hline\hline
      \ce{K2O}&0.23&\ce{Al2O3}&0.30&\ce{SiO2}&2.6\\\hline
      ZnO&0.27&&&\ce{B2O3}&0.8\\\hline
      CaO&0.5&&&&\\\hline\hline
    \end{tabular}
    \caption{A boron glaze to melt at 1100\degree
      C.}
    \label{tab:boronglazefrit}
  \end{table}
\end{center}
%-------------------------------------------------------------------------------
Initially we calculate the recipe as it was done for the unfritted glaze. We 
get the \ce{K2O} from potash feldspar. Borax cannot be used for boric oxide 
because no \ce{Na2O} is needed in the formula and so boric acid is required. We 
get the CaO from whiting and the rest of the materials will be kaolin, quartz 
and zinc oxide.

We now decide what material to include in the frit batch and what to include in 
the ball milling only. This is done according to the above rules. We need to 
include all the soluble boric acid. Along with that we can also include whiting 
and zinc oxide and some potash feldspar but not all because its \ce{Al2O3} will 
reduce the frit's fluidity.

A frit formula could be as shown in table~\ref{tab:boronglazefritformula}

%-------------------------------------------------------------------------------
\begin{center}
  \renewcommand{\arraystretch}{1.5}
  \begin{table}\centering
    \begin{tabular}{|c|c|c|c|c|c|}\hline
      \multicolumn{2}{|c|}{\textbf{Fluxes}}
      &\multicolumn{2}{|c|}{\textbf{Stabilizer}}
      &\multicolumn{2}{|c|}{\textbf{Glass Former}}\\\hline\hline
      \ce{K2O}&0.1&\ce{Al2O3}&0.1&\ce{SiO2}&1.6\\\hline
      ZnO&0.27&&&\ce{B2O3}&0.8\\\hline
      CaO&0.5&&&&\\\hline\hline
    \end{tabular}
    \caption{The formula of a boron glaze to melt at 1100\degree
      C.}
    \label{tab:boronglazefritformula}
  \end{table}
\end{center}
%-------------------------------------------------------------------------------
One problem still remains. When the frit melts, a large amount of \ce{H2O} and 
\ce{CO2} is lost. Thus loss does not influence the recipe if we weigh the raw 
frit materials, melt the frit and use all the melted frit in the glaze, adding 
the other material according to the original amount of raw frit. But it is much 
more practical to produce a large batch of frit at a time and later weigh the 
melted frit to produce smaller batches of glaze. We need to find out how much 
weight is lost.
%-------------------------------------------------------------------------------
\subsection{Frit Loss Calculation}
\subsubsection{Practical Loss}
The loss can be found simply by weighing the amount of melted frit that is 
produced from a batch of frit. 

For example:
\begin{itemize}
\item Raw frit batch weighs in total 500 kg.
\item After firing the (dry) frit weighs 280 kg.
\end{itemize}

$Loss~in~Percentage=(()500-280)/500)*100=44\%$

\subsubsection{Theoretical loss}
The loss can also be calculated based on the formula of the frit. On heating, 
whiting changes to calcium oxide.

\ce{CaCo3}+heat\ce{->CaO+CO2}
Only CaO enters the melted frit and we can calculate how much this weighs.

The MW of \ce{CaCo3} is 100, and that of CaO is 56: so loss is 44 parts. In 
percentage this is 44\%.

The number used to find the amount of oxide entering fusion is called the 
conversion factor, CF. Table~\ref{tab:fritconversionfactors} lists the 
conversion factor for the most common frit materials.
%-------------------------------------------------------------------------------
\begin{center}
  \renewcommand{\arraystretch}{1.5}
  \begin{table}\centering
    \begin{tabular}{|c|c|c|}\hline
      \textbf{Material}&\textbf{CF}&\textbf{\% Loss}\\\hline\hline
      Barium carbonate&0.777&22.3\\\hline
      Borax (crystal)&0.526&47.4\\\hline
      Boric acid&0.563&43.7\\\hline
      Dolomite&0.523&47.7\\\hline
      Kaolin&0.861&13.9\\\hline
      Lead carbonate (white)&0.863&13.7\\\hline
      Lead oxide (red)&0.977&2.3\\\hline
      Magnesium carbonate&0.478&52.2\\\hline
      Pearl ash&0.682&31.8\\\hline
      Soda ash&0.585&41.5\\\hline
      Soda crystals&0.217&78.3\\\hline
      Whiting&0.561&43.9\\\hline
    \end{tabular}
    \caption{Conversion factors for common oxides entering fusion in a frit.}
    \label{tab:fritconversionfactors}
  \end{table}
\end{center}
%-------------------------------------------------------------------------------
\subsubsection{Frit glaze example}
We can now calculate the loss of our frit with the conversion factors, as 
shown in table~\ref{tab:boronglazefritcf}.
%-------------------------------------------------------------------------------
\begin{center}
  \renewcommand{\arraystretch}{1.5}
  \begin{table}\centering
    \begin{tabular}{|c|c|c|c|}\hline
      \textbf{Material}&\textbf{Raw}&\textbf{CF}&\textbf{Melted}\\\hline\hline
      Potash feldspar&55.6&$=$&55.6\\\hline
      Whiting&50.0&$*0.561=$&28.1\\\hline
      Quartz&60.0&$=$&60.0\\\hline
      Zinc oxide&21.9&$=$&21.9\\\hline
      Boric acid&98.4&$*0.563=$&55.4\\\hline\hline
      \textbf{Total}&\textbf{285.9}&&\textbf{221.0}\\\hline
    \end{tabular}
    \caption{Using conversion factors to determine the loss of the frit in 
    table~\ref{tab:boronglazefritformula}.}
    \label{tab:boronglazefritcf}
  \end{table}
\end{center}
%-------------------------------------------------------------------------------
Theoretically we get only 77.3\% melted frit from our raw frit batch. We found 
that 286.3 parts raw frit equal 221.2 parts melted frit so finally we can 
establish our glaze recipe based on the melted frit. The final glaze recipe is 
shown in table~\ref{tab:boronglazefritfinal}.
%-------------------------------------------------------------------------------
\begin{center}
  \renewcommand{\arraystretch}{1.5}
  \begin{table}\centering
    \begin{tabular}{|c|c|c|}\hline
      \textbf{Material}&\textbf{Weight}&\textbf{Percent}\\\hline\hline
      Frit&221.0&69.9\%\\\hline
      Potash feldspar&72.3&22.9\%\\\hline
      Kaolin&18.1&5.7\%\\\hline
      Quartz&4.8&1.5\%\\\hline
    \end{tabular}
    \caption{The recipe for a boron-based frit to melt at 1100\degree C, after 
    compensating for loss.}
    \label{tab:boronglazefritfinal}
  \end{table}
\end{center}
%-------------------------------------------------------------------------------
\subsection{Glaze Recipe with Standard Frit}
Very often a ceramics producer gets the frit from a commercial supplier or 
wants to use only a few standard frits. Above we calculated a new frit based on 
the glaze formula. We will now calculate a glaze recipe from formula using a 
standard frit instead.

An example standard frit formula is shown in 
table~\ref{tab:standardfritformula}. 

We will try to use the frit in the glaze 
shown in table~\ref{tab:standardfritformulaglaze}. The calculation is done as 
with the unfritted glaze, and shown in 
table~\ref{tab:standardfritformulaglazeweight} First oxides are entered at the 
top of the table and we start to select materials to satisfy them. Before 
starting, we need to know the formula weight of the frit. In the appendix we 
get the MW of all the oxides and these we total.

The frit is entered in the calculation table like other materials with many 
oxides. The MP is selected according to the need of B2O3 It takes 0.8 MP of 
frit to get the needed 0.8 B2O3 and all the oxides listed in the frit formula 
are multiplied by this number and the results entered on the right of the table.

The final glaze recipe is shown in 
table~\ref{tab:standardfritformulaglazefinal}.

%-------------------------------------------------------------------------------
\begin{center}
  \renewcommand{\arraystretch}{1.5}
  \begin{table}\centering
    \begin{tabular}{|c|c|c|c|c|c|}\hline
      \multicolumn{2}{|c|}{\textbf{Fluxes}}
      &\multicolumn{2}{|c|}{\textbf{Stabilizer}}
      &\multicolumn{2}{|c|}{\textbf{Glass Former}}\\\hline\hline
      \ce{K2O}&0.26&\ce{Al2O3}&0.05&\ce{SiO2}&2.5\\\hline
      ZnO&0.13&&&\ce{B2O3}&1.0\\\hline
      CaO&0.61&&&&\\\hline\hline
    \end{tabular}
    \caption{A standard frit formula.}
    \label{tab:standardfritformula}
  \end{table}
\end{center}
%-------------------------------------------------------------------------------
\begin{center}
  \renewcommand{\arraystretch}{1.5}
  \begin{table}\centering
    \begin{tabular}{|c|c|c|c|c|c|}\hline
      \multicolumn{2}{|c|}{\textbf{Fluxes}}
      &\multicolumn{2}{|c|}{\textbf{Stabilizer}}
      &\multicolumn{2}{|c|}{\textbf{Glass Former}}\\\hline\hline
      \ce{K2O}&0.30&\ce{Al2O3}&0.40&\ce{SiO2}&3.5\\\hline
      ZnO&0.20&&&\ce{B2O3}&0.8\\\hline
      CaO&0.50&&&&\\\hline\hline
    \end{tabular}
    \caption{A glaze within which we want to use the standard frit formula in 
    table~\ref{tab:standardfritformula}.}
    \label{tab:standardfritformulaglaze}
  \end{table}
\end{center}
%-------------------------------------------------------------------------------
\begin{center}
  \renewcommand{\arraystretch}{1.5}
  \begin{table}\centering
    \begin{tabular}{|c|c|c|c|}\hline
      \textbf{Material}&\textbf{Calculation}&\textbf{Decimal}&\textbf{Percentage}
      \\\hline\hline
      \ce{K20}&$0.26*94=$&24.4\\\hline
      \ce{Na20}&$0.13*62=$&8.1\\\hline
      \ce{CaO}&$0.61*56=$&34.2\\\hline
      \ce{Al2O3}&$0.05*102=$&5.1\\\hline      
      \ce{SiO2}&$2.5*60=$&150\\\hline      
      \ce{B2O3}&$1.0*70=$&70\\\hline      
      \textbf{Frit Molecular Weight}&\textbf{291.8}\\\hline
      \textbf{Rounded}&&\textbf{292}\\\hline      
    \end{tabular}
    \caption{Converting molecular weights from      
      table~\ref{tab:standardfritformulaglaze} into percentages.}
    \label{tab:standardfritformulaglazeweight}
  \end{table}
\end{center}
%-------------------------------------------------------------------------------
\begin{center}
  \renewcommand{\arraystretch}{1.5}
  \begin{table}\centering
    \begin{tabular}{|c|c|c|}\hline
      \textbf{Material}&\textbf{Weight}&\textbf{Percent}\\\hline\hline
      Frit&233.6&61.7\%\\\hline
      Potash feldspar&51.2&13.5\%\\\hline
      Soda feldspar&50.3&13.3\%\\\hline
      Kaolin&42.7&11.2\%\\\hline
      Whiting&1&0.3\%\\\hline
    \end{tabular}
    \caption{The final recipe for a glaze using the standard frit shown in 
    table~\ref{tab:standardfritformula}.}
    \label{tab:standardfritformulaglazefinal}
  \end{table}
\end{center}
%-------------------------------------------------------------------------------
\section{Hints for Using Unknown Local Materials}
We have already discussed above calculating local materials by guessing their 
closest theoretical formula. This will usually give a good starting point for 
making line blends, which then can be used to get a working glaze or frit.

What do you do when you have a recipe or formula but do not know the analysis 
of your local materials and cannot get pure ones? Usually you can create a 
glaze using the formula or recipe as a starting point, but it is unlikely to 
match the description in the book.

The most common local materials are usually:
%-------------------------------------------------------------------------------
\subsection{Clays}
Common clays can be used in most glazes instead of kaolin, since they all 
contain \ce{Al2O3} and \ce{SiO2}. But they will have lower melting points and 
probably change the glaze color, since they will introduce \ce{K2O}, \ce{Na2O}, 
\ce{Fe2O3}, CaO, MgO and perhaps other fluxes. Probably the easiest way to work 
with them is simply to substitute directly for the kaolin, fire a sample and 
then use it as the basis for line blends to get a working glaze.
%-------------------------------------------------------------------------------
\subsection{Feldspars}
There are a tremendous number of different feldspars, all of which vary in the 
relative amounts of \ce{K2O}, \ce{Na2O}, CaO, MgO, \ce{Al2O3} and \ce{SiO2} 
they supply. This means that directly substituting feldspars will affect the 
melting point of the glaze, and possibly its color response. Try them out as 
direct substitutions, and then the result can be altered using line blends. If 
the new glaze seems underfired (dry surface), the fluxes can be increased. If 
it seems overfired (too fluid), the clay content can be increased.
%-------------------------------------------------------------------------------
\subsection{\ce{CaO} Sources}
Calcium is introduced into glazes from a large variety of raw materials: 
calcium carbonate, whiting, limestone, marble, seashells, coral, agricultural 
lime, etc. Usually, substituting will not make much difference, but again the 
result can be developed using line blends of the new material.
%-------------------------------------------------------------------------------
\subsection{Glass Cullet}
Glass cullet means waste glass, which can be used as the basis for cheap 
glazes. The best glass to use is window glass, which can usually be obtained 
free of charge or cheap from glass suppliers. Window glass consists of 
soda-lime-silica and can be used as a frit in glazes. It melts at about 
1100\degree C. With the addition of some flux and clay, it can be made into a 
low temperature glaze. However, because of its high CE, it will usually craze.
%-------------------------------------------------------------------------------
\subsection{Unknown Materials}
If you find new materials that are completely unknown, the easiest way to find 
out what they do is to first fire a small sample of the material alone, to see 
if it melts or not and what color it becomes. If it melts, it is a strong flux. 
If it does not melt, it may still be a flux. Check the test carefully to see if 
it has reacted with the clay body. If it develops a strong color, it will 
probably affect the glaze colour.

The material should also be tested by adding it to a known glaze recipe as a 
line blend.
%-------------------------------------------------------------------------------
