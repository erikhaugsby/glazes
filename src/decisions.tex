\chapter{Decisions}
As a ceramics entrepreneur, you must start by making decisions: what product? 
what temperature? how much technology? These decisions depend on your market, 
raw material and fuel availability. In industrialized countries, where 
everything is easily available, the decision will usually be based first on the 
market, and then the best combination of clay body, glazes and kiln can be 
decided on.

In developing countries, it is usually necessary to start by thinking about raw 
materials and fuel. Then the product can be selected.

Usually, it is easiest to use the same technology as other producers, as most 
of the problems will have already been solved. On the other hand, a new type of 
technology can capture a new market sector with no competition. However, a new 
technology may cause technical problems that a potter cannot solve without 
outside help.

Some typical questions for the entrepreneur to answer are given below.
%------------------------------------------------------------------------------
\section{Selecting Your Best Firing Temperature}
\begin{itemize}
\item Is high-firing clay available?

If so, it may be best to decide on high temperature ceramics (stoneware or 
porcelain), as producing reliable glazes will be easier.

\item Is only low-firing clay available?

If so, it will be necessary to select a low temperature system and to make 
frits or purchase ready-made ones.

\item Are ready-made glazes available?

If there is a reliable source of glazes nearby, a lot of trouble can be saved 
by using these.

\item What are the fuel constraints?

If only electric firing is available, then only low temperature systems will be 
practical. If oil or coal is available, the additional costs of using saggers 
should be compared with the cost of clean-burning fuels.
\end{itemize}
%------------------------------------------------------------------------------
\section{Market Factors}
Most producers decide to enter the ceramics sector because there is already a 
good market and not enough local supply or because they think they can create a 
market for products that are not yet common in their area.
%------------------------------------------------------------------------------
\begin{itemize}
\item What is the existing market?

For example, if there is already a good market for glazed white earthenware 
(perhaps imported), the potential producer will have to find out if he can 
produce similar products at competitive prices. If he wants to compete 
directly, he will have to take up the same clay/glaze/firing system.

\item Is there a possible new market?

On the other hand, it may be possible to produce a product with the same 
function, but using a less costly technology. For example, it may be possible 
to produce glazed red clay earthenware cheaper than the whiteware on the market 
and thus to create a new market.

\item Small-scale vs. large-scale

Large-scale ceramics industries are able to produce a large volume at a low 
profit margin. For this reason, it is difficult for the small producer to 
compete directly. The small producer has an advantage of flexibility - he can 
produce a variety of products on demand and thus can supply local customers 
with special requirements.

For example, the modern tile industry is mostly very large-scale and can supply 
very cheap tiles of a uniform quality. The small producer can never compete 
directly with this. However, there is a growing market for specialty tiles, 
with decorations or relief designs, which the large producers cannot make. Many 
customers are interested in small quantities of special decorative tiles made 
according to their own design, even if the price per square foot is higher than 
mass-produced tiles.

\item Ceramics substituted for products made from other materials

In some countries, products like glasses for drinking tea may be produced more 
cheaply in ceramics. Or cement sewage pipes and toilet pans may be replaced by 
longer-lasting, more hygienic ceramic products.
\end{itemize}
%------------------------------------------------------------------------------
\section{Strength Requirements}
%------------------------------------------------------------------------------
\begin{itemize}
\item Household items

Most common tableware items (cups, plates) can be made satisfactorily using 
either high or low temperature systems. Low temperature ceramics are more 
easily chipped and broken, but their low cost may be an advantage. High 
temperature products are stronger, and most hotels and restaurants will prefer 
them, unless the lower cost of earthenware makes up for the higher rate of 
breakage.

\item Electrical insulators

Low tension insulators, fuse holders (kit-kats) etc. do not have to be very 
strong, so can be made in the low temperature range. High tension insulators 
have special requirements for porosity and strength, so must necessarily be 
made at high temperature.

\item Tiles

Glazed tiles are most commonly produced at low temperatures, which gives them 
sufficient strength for wall and floor applications.

\item Cold climates

Ceramic products to be used outdoors in freezing temperatures have special 
requirements, because of damage that can come from water freezing inside the 
product and causing it to break. These products are generally made at high 
temperatures, which make it possible to control water absorption.
%------------------------------------------------------------------------------
\end{itemize}
%------------------------------------------------------------------------------
\section{Investment and Production Costs}
After considering the above decisions, the entrepreneur must then make an 
analysis of investment and production costs. These calculations are not easy to 
do, as the production of ceramics depends on so many complicated factors. For 
the new entrepreneur, it is important to start small and as simply as possible.

Low temperature systems usually require a lower initial investment, as kilns 
and burners will be cheaper. Fuel is usually the highest cost of production, 
and firing at low temperatures can save production costs. On the other hand, 
the cost of high temperature glazes is lower, as it is not necessary to use 
expensive frits.

In preparing a scheme for a new business, it is best to get help from a 
ceramics expert' who can help to figure out the comparative costs of the 
various options. Besides the usual overhead costs, it is necessary to consider:
%------------------------------------------------------------------------------
\begin{itemize}
\item cost of clay body
\item cost of glaze
\item labor costs in production
\item capital investment for equipment
\item fuel costs
\item working capital requirements.
\end{itemize}
%------------------------------------------------------------------------------