\chapter{Introduction and Scope}
There are already many books in the world on the subject of ceramic glazes. So 
the obvious question is: why yet another book on the subject? The authors have 
worked together for several years in a ceramics development project in Nepal, 
which is based on using local raw materials and resources. There are few 
existing books which offer much help in this area, especially working in the 
low temperature range from 900\degree C--1100\degree C, where lead glazes 
have been the tradition but which now, with greater understanding of health 
hazards, need to be replaced with lead-free glazes. This book is intended to 
provide practical information for ceramists working in developing countries, 
with little access to the prepared and controlled glaze materials available in 
industrialized nations.

Glazes are at one and the same time the area of most fascination and most 
difficulty for potters. Most potters have little inclination or time to devote 
to developing glazes, faced as they are by the daily need to produce for the 
market. However, there often are times when familiar glazes suddenly stop 
working correctly or special glazes are required for customers. This book 
offers guidelines for developing and altering glazes, understanding where 
problems with glazes come from, and standard procedures for testing and 
developing glazes when there is no laboratory equipment available. It has been 
written for potters who have little knowledge of chemistry and mathematics.
%-------------------------------------------------------------------------------
\section{Glaze-Making: Using Local Materials As Far As Possible}
Most small producers of glazed ceramics will use glazes that are prepared by a 
company specializing in supplying industry. However, these glazes are often 
unreliable, as big companies tend to serve large-scale producers and have 
little interest in the special glazes needed by small industries. For that 
reason, the small producer is often forced to rely on his own glaze production, 
with little or no laboratory equipment available. Additionally, the small 
producer does not usually have access to raw materials at reasonable prices, so 
he must use locally available raw materials that do not have an accurate 
chemical analysis.
%-------------------------------------------------------------------------------
\section{Glaze and Clay Systems}
The producer must think carefully before starting production. When a particular 
glaze is wanted, it must work with the available clay body, production system 
and firing system. 

For example, if you only have low-temperature red clay available, your glazes 
must work at around 1050\degree C. If you only have coal available for firing, 
you must make sure that it will work for your product. The following chapters 
provide information which will help you to make these decisions.
%-------------------------------------------------------------------------------