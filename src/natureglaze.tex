\chapter{The Nature of Glazes}
%-------------------------------------------------------------------------------
\section{Glass and Glaze, the Benefit of Glaze}
Glass is a useful material that has been known for thousands of years. It can 
be produced in many different shapes for many purposes, and it has many useful 
qualities: it is transparent, hard, resistant to chemicals, and can have many 
colors.

Glaze is a special type of glass, made for coating ceramic products. Whereas 
glass is suitable for forming into bottles or windows, glaze is different 
because it is applied on a ceramic surface and must form a hard, durable 
coating after being melted in the kiln. It must not run off the product and 
must stay on the product after firing without cracking.
%-------------------------------------------------------------------------------
\section{Glaze-Making is Difficult}
Because glaze before firing looks nothing like the finished product and because 
we are not able to directly understand what happens when glaze melts at high 
temperatures, making glazes is very difficult. We must try to understand which 
materials melt at certain temperatures and what happens when materials are 
combined. It requires a lot of direct experience before you start to understand 
causes and effects. In this way it is like cooking: we are familiar with 
cooking because we know the raw materials, and by trial and error we have a 
good idea of what the finished meal will be like. However, imagine that you are 
in a foreign country with unfamiliar food in the market and you want to make a 
meal: How do you start? The best way is with a cookbook full of recipes and a 
local friend to tell you if the result is correct or not. This book is intended 
as a cookbook for the independent potter.

Although by just reading this book and experimenting with it you will probably 
be able to make glazes after some time, there is no substitute for learning 
about glazes from an experienced teacher, who can save you a lot of time by 
guiding you in proven directions.
%-------------------------------------------------------------------------------
\section{History of Glazes}
Unglazed ceramics have been in existence for over 10,000 years. It has only 
been in the last 2000 years that there have been glazed ceramics and only in 
the last 100 years that a scientific approach to glaze making was developed. 
For that reason, glazes still occupy a mysterious area somewhere between 
science and magic.

The first glazes were probably invented in middle eastern countries, where 
there naturally exist deposits of sodium and potassium compounds (soda ash and 
pearl ash) that melt at low temperatures (800\degree --1000\degree C). By 
chance, early potters discovered that some clays when put in the fire developed 
a shiny surface. These self-glazing clays are known as ``Egyptian paste''. They 
are not very useful for making household items, being difficult to form.
The next step was to develop these substances so that they could be applied to 
the surface of pottery clay in order to give it the desirable qualities of a 
hard, shiny, easy-to-clean and durable surface. Because early potters did not 
have the technology to reach high firing temperatures, they had to use 
materials with low melting points, mainly sodium, potassium and lead compounds. 

Glaze development had to be done by trial and error, since these early potters 
had no idea of chemistry. This took a lot of time and effort, and naturally 
successful glazes were closely guarded secrets. These early glazes were often 
soft and not durable, and had problems such as cracking and eventually falling 
off the pot. Additionally, glazes based on lead were poisonous both for the 
potters who worked with them and for users.

It was only when potters learned to reach high temperatures that truly 
permanent ceramics were developed. There are many more common chemicals and 
minerals that melt above 1100\degree C to form glazes, and clay that is fired 
to these high temperatures is also much stronger and resistant to water.
%-------------------------------------------------------------------------------
\section{Glaze Classification}
Although there are many different ways to classify glazes, the simplest way to 
understand them is according to the firing temperature. The useful range of 
temperatures for glaze melting is from 900\degree --1300\degree C. In this 
book, we talk about two different categories of glaze:
%-------------------------------------------------------------------------------
\begin{itemize}
\item low temperature from 900-1100\degree C, called earthenware
\item high temperature from 1100-1300\degree C, called stoneware
\end{itemize} 
%-------------------------------------------------------------------------------
These two categories are used because they require different raw materials as 
the main ingredients of the glaze.
%------------------------------------------------------------------------------