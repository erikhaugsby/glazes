\chapter{Health and Safety}
As in all other types of industries, precautions are needed to avoid health 
hazards to the ceramics workers.
%------------------------------------------------------------------------------
\section{Machinery}
Moving parts of machinery used in the workshop should be enclosed to prevent 
hands, clothing or hair being caught in them. The belts and gears of ball 
mills, hammer mills etc. are especially dangerous.

Place the electrical switch next to machines, where the operator can reach it.
%------------------------------------------------------------------------------
\section{Dust}
Workers in the ceramics industry are constantly exposed to dust. Inhalation 
of-dust from clay materials and quartz will cause silicosis. This is an 
incurable lung disease. The dangerous dust is so fine it cannot be seen.

The workshop floor should be cleaned regularly by scrubbing it with water. Dry 
sweeping should never take place. If it is not possible to wash the floors they 
can be swept after spreading wet or better still oiled sawdust. Tables, shelves 
and other surfaces collecting dust should be cleaned with a wet sponge at least 
once a week.

Dry blending of glaze and clay materials should be avoided. If it is done, the 
worker must wear a dust mask.

If the climate allows it, keep doors and windows open. Good ventilation will 
reduce the dust hazard.
%------------------------------------------------------------------------------
\section{Toxic Materials}
\subsection{Hazards to Workers}
Some glaze materials are directly poisonous if eaten or inhaled. The effect is 
not immediate but accumulates in the body over the years. The most dangerous 
are raw lead materials. Lead compounds should only be used as a frit. Other 
toxic materials are:
%------------------------------------------------------------------------------
\begin{itemize}
\item Antimony oxide
\item Barium carbonate
\item Cadmium compounds (in color pigment)
\item Chromium dioxide
\item Cobalt oxide and carbonate
\item Copper oxide and carbonate
\item Nickel oxide
\item Zinc oxide
\end{itemize}
%------------------------------------------------------------------------------
Preventive rules are:
%------------------------------------------------------------------------------
\begin{itemize}
\item Wear a dust mask when dry mixing the materials.
\item Wash hands after working with these materials.
\item Wear special clothing only for working.
\item Never eat, drink or smoke in the workshop.
\end{itemize}
%------------------------------------------------------------------------------
\subsection{Hazards to Crockery Users}
The main danger for users of crockery is the release of lead from glazes. This 
may happened if the glaze contains free lead and the glaze is used for storing 
acidic food. Glazes made with lead frits may be perfectly safe but it depends 
very much on the composition of the glaze. Unless your crockery can be checked 
regularly by a chemical laboratory, it is safer not to use lead glazes for 
items meant for food.
%------------------------------------------------------------------------------