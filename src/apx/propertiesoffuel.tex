\chapter{Properties of Fuels}
\label{propertiesoffuel}
The average properties of solid fuels used for firing ceramic kilns.

Heat or calorific value is measured in calories per gram of fuel. One calorie 
is the heat required to heat 1 gram of water 1\degree C.

Gross calorific value is the heat that theoretically can be obtained, whereas 
net value is what is normally obtained when firing a kiln. Both values are 
included for comparison with other fuels.
%-------------------------------------------------------------------------------
\begin{landscape}
  \begin{center}

    \begin{table}\centering
    \renewcommand{\arraystretch}{1.1}      
    \begin{tabular}{|c|c||c|c|c|c|c|}\hline
  \multicolumn{7}{|c|}{\textbf{Fuel}}\\\hline  
  &&\textbf{Wood}&\textbf{Peat}&\textbf{Lignite}&\textbf{Bituminous 
    Coal}&\textbf{Charcoal}\\\hline\hline
  Moisture content as found&\%&25--50&90&50&2&\\\hline
  Moisture content at firing&\%&10--15&15--20&15&2&2\\\hline
  Volatile matters&\%&80&65&50&30&10\\\hline
  Fixed carbon&\%&20&30&45&65&89\\\hline
  Ash&\%&trace&5&5&5&1\\\hline
  \multicolumn{7}{|c|}{\textbf{Chemical Analysis}}\\\hline
  Carbon&\%&50.0&57.5&70.0&86.0&93.0\\\hline
  Hydrogen&\%&6.0&5.5&5.0&5.5&2.5\\\hline
  Oxygen&\%&43.0&35.0&23.00&6.0&3.0\\\hline
  Nitrogen \& Sulphur&\%&1.0&2.0&2.0&2.5&1.5\\\hline
  \multicolumn{7}{|c|}{\textbf{Calorific Value (cal/g)}}\\\hline
  dry fuel&gross&4450&5000&6400&8600&8300\\\hline
  &net&4130&4710&6140&8310&8170\\\hline
  normal fuel&gross&3780&3800&5170&8000&8050\\\hline
  &net&3420&3460&4870&7720&7910\\\hline
\end{tabular}
\caption{Average properties of solid fuels.}
\label{tab:propertiesoffuelsolid}    \end{table}
  \end{center}
\end{landscape}
%-------------------------------------------------------------------------------
  \begin{center}
    \renewcommand{\arraystretch}{1.5}
    \begin{table}\centering
      \begin{tabular}{|c|c||c|c|c|}\hline
    &&\textbf{Specific gravity}& \textbf{Ash \%}&\textbf{cal/g} \\\hline\hline
    Hardwood: & Ash   & .74                       & .6              & 
    4450           \\\hline
    & Beech & .68                       & .6              & 4500           
    \\\hline
    & Oak   & .83                       & .4              & 4360           
    \\\hline
    Softwood: & Fir   & .45                       & .3              & 
    4770           \\\hline
    & Pine  & .48                       & .4              & 4820           
    \\\hline
    & Elm   & .56                       & .5              & 4470\\\hline
  \end{tabular}
\caption{Average properties of dry wood.}
\label{tab:propertiesoffuelwood}
\end{table}
\end{center}

%-------------------------------------------------------------------------------
\begin{landscape}
  \begin{center}
    \renewcommand{\arraystretch}{1.5}
    \begin{table}\centering
  \begin{tabular}{|c||c|c|c|c|c|}\hline
        \multicolumn{6}{|c|}{\textbf{Fuel}}\\ \hline
    & \textbf{Waste oil} & \textbf{Heavy fuel oil} & 
    \textbf{Medium fuel oil} & \textbf{Light fuel oil} & \textbf{Kerosene} 
    \\\hline 
    \hline
    \textbf{Specific gravity}     & 0.9 - 1            & 1.1 - 
    0.94              & 0.93 - 0.91              & 0.9 - 0.81              & 
    0.78              \\ \hline
    \textbf{Flash point \degree C}        & 250                & 
    200                     & 150                      & 
    105                     & 55                \\ \hline
    \textbf{Viscosity}              & very high          & 
    high                    & medium                   & 
    low                     & very low          \\ \hline
    \multicolumn{6}{|c|}{\textbf{Calorific Value}}\\ \hline
    \textbf{Gross} & 10300              & 
    10055                   & 10130                    & 
    10300                   & 11100             \\ \hline
     \textbf{Net}   & 9480               & 
    9536                    & 9695                     
    &                        10130 & 11100                  \\ \hline
  \end{tabular}
\caption{Average properties of liquid fuels.}
\label{tab:propertiesoffuelliquid}
\end{table}
\end{center}
\end{landscape}
%-------------------------------------------------------------------------------