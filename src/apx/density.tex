\chapter{Density}
\label{sec:density}
Specific gravity (SG) of a material, a mixture of materials or a clay slip is 
expressed as how many times it is heavier than the same amount of water, i.e. 
how many kg per 1 liter volume or gram per $cm^3$. 

Density is the weight per volume unit and in the metric system this equals 
specific gravity (g/cc or kg/l) but in many countries slip densities are still 
measured in ounces per pint.

The density of a clay slip is found by weighing 1 liter of the slip. If it 
weighs 1.6 kg the slip has a density of 1.6.
%-------------------------------------------------------------------------------